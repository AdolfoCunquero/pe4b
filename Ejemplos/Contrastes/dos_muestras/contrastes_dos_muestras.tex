% Options for packages loaded elsewhere
\PassOptionsToPackage{unicode}{hyperref}
\PassOptionsToPackage{hyphens}{url}
%
\documentclass[
  ignorenonframetext,
]{beamer}
\usepackage{pgfpages}
\setbeamertemplate{caption}[numbered]
\setbeamertemplate{caption label separator}{: }
\setbeamercolor{caption name}{fg=normal text.fg}
\beamertemplatenavigationsymbolsempty
% Prevent slide breaks in the middle of a paragraph
\widowpenalties 1 10000
\raggedbottom
\setbeamertemplate{part page}{
  \centering
  \begin{beamercolorbox}[sep=16pt,center]{part title}
    \usebeamerfont{part title}\insertpart\par
  \end{beamercolorbox}
}
\setbeamertemplate{section page}{
  \centering
  \begin{beamercolorbox}[sep=12pt,center]{part title}
    \usebeamerfont{section title}\insertsection\par
  \end{beamercolorbox}
}
\setbeamertemplate{subsection page}{
  \centering
  \begin{beamercolorbox}[sep=8pt,center]{part title}
    \usebeamerfont{subsection title}\insertsubsection\par
  \end{beamercolorbox}
}
\AtBeginPart{
  \frame{\partpage}
}
\AtBeginSection{
  \ifbibliography
  \else
    \frame{\sectionpage}
  \fi
}
\AtBeginSubsection{
  \frame{\subsectionpage}
}
\usepackage{lmodern}
\usepackage{amssymb,amsmath}
\usepackage{ifxetex,ifluatex}
\ifnum 0\ifxetex 1\fi\ifluatex 1\fi=0 % if pdftex
  \usepackage[T1]{fontenc}
  \usepackage[utf8]{inputenc}
  \usepackage{textcomp} % provide euro and other symbols
\else % if luatex or xetex
  \usepackage{unicode-math}
  \defaultfontfeatures{Scale=MatchLowercase}
  \defaultfontfeatures[\rmfamily]{Ligatures=TeX,Scale=1}
\fi
\usetheme[]{Warsaw}
% Use upquote if available, for straight quotes in verbatim environments
\IfFileExists{upquote.sty}{\usepackage{upquote}}{}
\IfFileExists{microtype.sty}{% use microtype if available
  \usepackage[]{microtype}
  \UseMicrotypeSet[protrusion]{basicmath} % disable protrusion for tt fonts
}{}
\makeatletter
\@ifundefined{KOMAClassName}{% if non-KOMA class
  \IfFileExists{parskip.sty}{%
    \usepackage{parskip}
  }{% else
    \setlength{\parindent}{0pt}
    \setlength{\parskip}{6pt plus 2pt minus 1pt}}
}{% if KOMA class
  \KOMAoptions{parskip=half}}
\makeatother
\usepackage{xcolor}
\IfFileExists{xurl.sty}{\usepackage{xurl}}{} % add URL line breaks if available
\IfFileExists{bookmark.sty}{\usepackage{bookmark}}{\usepackage{hyperref}}
\hypersetup{
  pdftitle={Ejemplos contrastes de hipótesis de dos muestras},
  pdfauthor={Ricardo Alberich, Juan Gabriel Gomila y Arnau Mir},
  hidelinks,
  pdfcreator={LaTeX via pandoc}}
\urlstyle{same} % disable monospaced font for URLs
\newif\ifbibliography
\usepackage{color}
\usepackage{fancyvrb}
\newcommand{\VerbBar}{|}
\newcommand{\VERB}{\Verb[commandchars=\\\{\}]}
\DefineVerbatimEnvironment{Highlighting}{Verbatim}{commandchars=\\\{\}}
% Add ',fontsize=\small' for more characters per line
\usepackage{framed}
\definecolor{shadecolor}{RGB}{248,248,248}
\newenvironment{Shaded}{\begin{snugshade}}{\end{snugshade}}
\newcommand{\AlertTok}[1]{\textcolor[rgb]{0.94,0.16,0.16}{#1}}
\newcommand{\AnnotationTok}[1]{\textcolor[rgb]{0.56,0.35,0.01}{\textbf{\textit{#1}}}}
\newcommand{\AttributeTok}[1]{\textcolor[rgb]{0.77,0.63,0.00}{#1}}
\newcommand{\BaseNTok}[1]{\textcolor[rgb]{0.00,0.00,0.81}{#1}}
\newcommand{\BuiltInTok}[1]{#1}
\newcommand{\CharTok}[1]{\textcolor[rgb]{0.31,0.60,0.02}{#1}}
\newcommand{\CommentTok}[1]{\textcolor[rgb]{0.56,0.35,0.01}{\textit{#1}}}
\newcommand{\CommentVarTok}[1]{\textcolor[rgb]{0.56,0.35,0.01}{\textbf{\textit{#1}}}}
\newcommand{\ConstantTok}[1]{\textcolor[rgb]{0.00,0.00,0.00}{#1}}
\newcommand{\ControlFlowTok}[1]{\textcolor[rgb]{0.13,0.29,0.53}{\textbf{#1}}}
\newcommand{\DataTypeTok}[1]{\textcolor[rgb]{0.13,0.29,0.53}{#1}}
\newcommand{\DecValTok}[1]{\textcolor[rgb]{0.00,0.00,0.81}{#1}}
\newcommand{\DocumentationTok}[1]{\textcolor[rgb]{0.56,0.35,0.01}{\textbf{\textit{#1}}}}
\newcommand{\ErrorTok}[1]{\textcolor[rgb]{0.64,0.00,0.00}{\textbf{#1}}}
\newcommand{\ExtensionTok}[1]{#1}
\newcommand{\FloatTok}[1]{\textcolor[rgb]{0.00,0.00,0.81}{#1}}
\newcommand{\FunctionTok}[1]{\textcolor[rgb]{0.00,0.00,0.00}{#1}}
\newcommand{\ImportTok}[1]{#1}
\newcommand{\InformationTok}[1]{\textcolor[rgb]{0.56,0.35,0.01}{\textbf{\textit{#1}}}}
\newcommand{\KeywordTok}[1]{\textcolor[rgb]{0.13,0.29,0.53}{\textbf{#1}}}
\newcommand{\NormalTok}[1]{#1}
\newcommand{\OperatorTok}[1]{\textcolor[rgb]{0.81,0.36,0.00}{\textbf{#1}}}
\newcommand{\OtherTok}[1]{\textcolor[rgb]{0.56,0.35,0.01}{#1}}
\newcommand{\PreprocessorTok}[1]{\textcolor[rgb]{0.56,0.35,0.01}{\textit{#1}}}
\newcommand{\RegionMarkerTok}[1]{#1}
\newcommand{\SpecialCharTok}[1]{\textcolor[rgb]{0.00,0.00,0.00}{#1}}
\newcommand{\SpecialStringTok}[1]{\textcolor[rgb]{0.31,0.60,0.02}{#1}}
\newcommand{\StringTok}[1]{\textcolor[rgb]{0.31,0.60,0.02}{#1}}
\newcommand{\VariableTok}[1]{\textcolor[rgb]{0.00,0.00,0.00}{#1}}
\newcommand{\VerbatimStringTok}[1]{\textcolor[rgb]{0.31,0.60,0.02}{#1}}
\newcommand{\WarningTok}[1]{\textcolor[rgb]{0.56,0.35,0.01}{\textbf{\textit{#1}}}}
\usepackage{longtable,booktabs}
\usepackage{caption}
% Make caption package work with longtable
\makeatletter
\def\fnum@table{\tablename~\thetable}
\makeatother
\usepackage{graphicx}
\makeatletter
\def\maxwidth{\ifdim\Gin@nat@width>\linewidth\linewidth\else\Gin@nat@width\fi}
\def\maxheight{\ifdim\Gin@nat@height>\textheight\textheight\else\Gin@nat@height\fi}
\makeatother
% Scale images if necessary, so that they will not overflow the page
% margins by default, and it is still possible to overwrite the defaults
% using explicit options in \includegraphics[width, height, ...]{}
\setkeys{Gin}{width=\maxwidth,height=\maxheight,keepaspectratio}
% Set default figure placement to htbp
\makeatletter
\def\fps@figure{htbp}
\makeatother
\setlength{\emergencystretch}{3em} % prevent overfull lines
\providecommand{\tightlist}{%
  \setlength{\itemsep}{0pt}\setlength{\parskip}{0pt}}
\setcounter{secnumdepth}{-\maxdimen} % remove section numbering
\usepackage{amsmath,color,array,booktabs,algorithm2e,eurosym} 
\newcommand\blue[1]{\textcolor{blue}{#1}}
\newcommand\red[1]{\textcolor{red}{#1}}
\setbeamertemplate{navigation symbols}{}
\setbeamertemplate{footline}[page number]

\title{Ejemplos contrastes de hipótesis de dos muestras}
\author{Ricardo Alberich, Juan Gabriel Gomila y Arnau Mir}
\date{}

\begin{document}
\frame{\titlepage}

\begin{frame}{Comparación de las medias de las ventas con dos
estrategias de marketing}
\protect\hypertarget{comparaciuxf3n-de-las-medias-de-las-ventas-con-dos-estrategias-de-marketing}{}
\textbf{Ejemplo}

Un jefe de marketing quiere evaluar la eficacia de dos estrategias de
ventas \(M_1\) y \(M_2\). Se quiere averiguar si las vetas medias en
euros de cada campaña son iguales.

Para ello se han realizado dos muestras seleccionando al azar un
conjunto de clientes para la estrategia \(M_1\) y otro para la \(M_2\).

Desconocemos las dos desviaciones típicas poblaciones \(\sigma_1\) y
\(\sigma_2\) de cada una de las valoraciones de la campaña.
\end{frame}

\begin{frame}{Comparación de las medias de las ventas con dos
estrategias de marketing}
\protect\hypertarget{comparaciuxf3n-de-las-medias-de-las-ventas-con-dos-estrategias-de-marketing-1}{}
Las dos muestras se han seleccionado de forma independientes y sus
tamaños son \(n_1=60\) y \(n_2=50\) respectivamente.

La variable que se mide es el número de unidades vendidas

Las medias y las desviaciones típicas de las ventas en euros son: \[
\overline{X}_1= 101.78 ,\  \overline{X}_2=105.38,\ 
\widetilde{S}_1=6.2,\  \widetilde{S}_2=5.57.
\]
\end{frame}

\begin{frame}{Comparación de las medias de las ventas con dos
estrategias de marketing}
\protect\hypertarget{comparaciuxf3n-de-las-medias-de-las-ventas-con-dos-estrategias-de-marketing-2}{}
El contraste a realizar es el siguiente: \[
\left\{\begin{array}{l}
H_0:\mu_1=\mu_2,\\
H_1:\mu_1< \mu_2,
\end{array}\right.
\Longleftrightarrow
\left\{\begin{array}{l}
H_0:\mu_1-\mu_2=0,\\
H_1:\mu_1- \mu_2<0,
\end{array}\right.
\]

donde \(\mu_1\) y \(\mu_2\) representan las ventas medias en euros para
cada una de las dos campañas de marketing \(M_1\) y \(M_2\).

Consideremos los dos casos anteriores:

\begin{itemize}
\tightlist
\item
  Caso 1: Suponemos \(\sigma_1=\sigma_2\).
\end{itemize}

El \textbf{estadístico de contraste} es:

\[
T=\frac{\overline{X}_1-\overline{X}_2}
{\sqrt{(\frac1{n_1}+\frac1{n_2})\cdot 
\frac{((n_1-1)\widetilde{S}_1^2+(n_2-1)\widetilde{S}_2^2)}
{(n_1+n_2-2)}}}\sim t_{60+50-2}=t_{108},$
cuyo valor, usando los valores correspondientes de las muestras, será:
$t_0=\frac{101.78-105.38}{\sqrt{(\frac1{60}+\frac1{50})\frac{(59\cdot 6.2^2+49\cdot 5.57^2)}{108}}}=-3.174.
\]
\end{frame}

\begin{frame}{Comparación de las medias de las ventas con dos
estrategias de marketing}
\protect\hypertarget{comparaciuxf3n-de-las-medias-de-las-ventas-con-dos-estrategias-de-marketing-3}{}
El \(p\)-valor será, en este caso: \(P(t_{108}<-3.174)\approx 0.001,\)
valor muy pequeño.

La decisión que tomamos, por tanto, es rechazar la hipótesis de que son
iguales, en favor de que los estudiantes del grado \(G_1\) tardan menos
tiempo en realizar la tarea que los estudiantes del grado \(G_2\).
\end{frame}

\begin{frame}{Comparación de las medias de las ventas con dos
estrategias de marketing caso varianzas distintas}
\protect\hypertarget{comparaciuxf3n-de-las-medias-de-las-ventas-con-dos-estrategias-de-marketing-caso-varianzas-distintas}{}
Consideremos ahora el otro caso:

\begin{itemize}
\tightlist
\item
  Caso 2: Suponemos \(\sigma_1\neq \sigma_2\).
\end{itemize}

El \textbf{estadístico de contraste} será, en este caso:
\(T=\frac{\overline{X}_1-\overline{X}_2}{\sqrt{\frac{\widetilde{S}_1^2}{n_1}+\frac{\widetilde{S}_2^2}{n_2}}}\sim t_f\)
donde \[
f=\left\lfloor\frac{ \left( \frac{6.2^2}{60}+\frac{6.2^2}{50}\right)^2}
{\frac1{59}\left(\frac{6.2^2}{60}\right)^2+\frac1{49}\left(\frac{5.57^2}{50}\right)^2}\right\rfloor -2
=\lfloor 107.36=107.364576.
\]
\end{frame}

\begin{frame}{Ejemplo}
\protect\hypertarget{ejemplo}{}
El valor que toma el estadístico anterior será: \[
t_0=\frac{101.78-105.38}{\sqrt{\frac{6.2^2}{60}+\frac{5.57^2}{50}}}=-3.206.
\]

El \textbf{\(p\)-valor} del contraste será:
\(P(t_{105}\leq -3.206)= 0.001,\) valor muy pequeño.

La decisión que tomamos en este caso es la misma que en el caso
anterior: rechazar la hipótesis de que los tiempos de ejecución son
iguales, en favor de que los alumnos del grado \(G_1\) tardan menos
tiempo en realizar la tarea que los alumnos del grado \(G_2\).

La decisión final, al haber decidido lo mismo en los dos casos, será
concluir que los alumnos del grado \(G_1\) tardan menos tiempo en
realizar la tarea que los alumnos del grado \(G_2\).
\end{frame}

\begin{frame}[fragile]{Contrastes para dos medias independientes en
\texttt{R}: función \texttt{t.test}}
\protect\hypertarget{contrastes-para-dos-medias-independientes-en-r-funciuxf3n-t.test}{}
Recordemos la sintaxis básica de la función \texttt{t.test} es

\begin{Shaded}
\begin{Highlighting}[]
\KeywordTok{t.test}\NormalTok{(x, y, }\DataTypeTok{mu=}\NormalTok{..., }\DataTypeTok{alternative=}\NormalTok{..., }\DataTypeTok{conf.level=}\NormalTok{..., }\DataTypeTok{paired=}\NormalTok{..., }
       \DataTypeTok{var.equal=}\NormalTok{..., }\DataTypeTok{na.omit=}\NormalTok{...)}
\end{Highlighting}
\end{Shaded}

donde los nuevos parámetros para realizar un contraste de dos medias
independientes son:

\begin{itemize}
\item
  \texttt{x} es el vector de datos de la primera muestra.
\item
  \texttt{y} es el vector de datos de la segunda muestra.
\end{itemize}
\end{frame}

\begin{frame}[fragile]{Contraste de \(\mu\) de normal con \(\sigma\)
desconocida en \texttt{R}: función \texttt{t.test}}
\protect\hypertarget{contraste-de-mu-de-normal-con-sigma-desconocida-en-r-funciuxf3n-t.test}{}
\begin{itemize}
\item
  Podemos sustituir los vectores \texttt{x} e \texttt{y} por una fórmula
  \texttt{variable1\textasciitilde{}variable2} que indique que separamos
  la variable numérica \texttt{variable1} en dos vectores definidos por
  los niveles de un factor \texttt{variable2} de dos niveles (o de otra
  variable asimilable a un factor de dos niveles, como por ejemplo una
  variable numérica que solo tome dos valores diferentes).
\item
  Parámetro \texttt{alternative}:

  \begin{itemize}
  \tightlist
  \item
    Si llamamos \(\mu_x\) y \(\mu_y\) a las medias de las poblaciones de
    las que hemos extraído las muestras \(x\) e \(y\), respectivamente,
    entonces \texttt{"two.sided"} representa la hipótesis alternativa
    \(H_1: \mu_x \neq \mu_y\); \texttt{"less"} indica que la hipótesis
    alternativa es \(H_1: \mu_x< \mu_y\); y \texttt{"greater"}, que la
    hipótesis alternativa es \(H_1: \mu_x> \mu_y\).
  \end{itemize}
\end{itemize}
\end{frame}

\begin{frame}[fragile]{Contraste de \(\mu\) de normal con \(\sigma\)
desconocida en \texttt{R}: función \texttt{t.test}}
\protect\hypertarget{contraste-de-mu-de-normal-con-sigma-desconocida-en-r-funciuxf3n-t.test-1}{}
\begin{itemize}
\tightlist
\item
  El parámetro \texttt{var.equal} solo lo tenemos que especificar si
  llevamos a cabo un contraste de dos medias usando muestras
  independientes, y en este caso sirve para indicar si queremos
  considerar las dos varianzas poblacionales iguales (igualándolo a
  TRUE) o diferentes (igualándolo a FALSE, que es su valor por defecto).
\end{itemize}
\end{frame}

\begin{frame}[fragile]{Ejemplo}
\protect\hypertarget{ejemplo-1}{}
\textbf{Ejercicio}

Imaginemos ahora que nos planteamos si la media de la longitud del
pétalo es la misma para las flores de las especies setosa y versicolor.

Para ello seleccionamos una muestra de tamaño 40 flores para cada
especie:

\begin{Shaded}
\begin{Highlighting}[]
\KeywordTok{set.seed}\NormalTok{(}\DecValTok{45}\NormalTok{)}
\NormalTok{flores.elegidas.setosa =}\StringTok{ }\KeywordTok{sample}\NormalTok{(}\DecValTok{1}\OperatorTok{:}\DecValTok{50}\NormalTok{,}\DecValTok{40}\NormalTok{,}\DataTypeTok{replace=}\OtherTok{TRUE}\NormalTok{)}
\NormalTok{flores.elegidas.versicolor =}\StringTok{ }\KeywordTok{sample}\NormalTok{(}\DecValTok{51}\OperatorTok{:}\DecValTok{100}\NormalTok{,}\DecValTok{40}\NormalTok{,}\DataTypeTok{replace=}\OtherTok{TRUE}\NormalTok{)}
\end{Highlighting}
\end{Shaded}

Las muestras serán las siguientes:

\begin{Shaded}
\begin{Highlighting}[]
\NormalTok{muestra.setosa =}\StringTok{ }\NormalTok{iris[flores.elegidas.setosa,]}
\NormalTok{muestra.versicolor =}\StringTok{ }\NormalTok{iris[flores.elegidas.versicolor,]}
\end{Highlighting}
\end{Shaded}
\end{frame}

\begin{frame}[fragile]{Ejemplo}
\protect\hypertarget{ejemplo-2}{}
El contraste planteado se realiza de la forma siguiente:

\begin{Shaded}
\begin{Highlighting}[]
\KeywordTok{t.test}\NormalTok{(muestra.setosa}\OperatorTok{$}\NormalTok{Petal.Length,muestra.versicolor}\OperatorTok{$}\NormalTok{Petal.Length,}
       \DataTypeTok{alternative=}\StringTok{"two.sided"}\NormalTok{)}
\end{Highlighting}
\end{Shaded}

\begin{verbatim}
## 
##  Welch Two Sample t-test
## 
## data:  muestra.setosa$Petal.Length and muestra.versicolor$Petal.Length
## t = -42.766, df = 49.953, p-value < 2.2e-16
## alternative hypothesis: true difference in means is not equal to 0
## 95 percent confidence interval:
##  -2.913186 -2.651814
## sample estimates:
## mean of x mean of y 
##    1.4075    4.1900
\end{verbatim}
\end{frame}

\begin{frame}{Ejemplo}
\protect\hypertarget{ejemplo-3}{}
El contraste realizado es de dos muestras independientes: \[
\left.
\begin{array}{ll}
H_0: & \mu_{{setosa}} =\mu_{{versicolor}}, \\
H_1: & \mu_{{setosa}} \neq \mu_{{versicolor}},
\end{array}
\right\}
\] donde \(\mu_{{setosa}}\) representa la media de la longitud del
pétalo de las flores de la especie setosa y \(\mu_{{versicolor}}\), la
media de la longitud del pétalo de las flores de la especie versicolor.

El p-valor del contraste ha sido pràcticamente cero, lo que nos hace
concluir que tenemos evidencias suficientes para concluir que las medias
de la longitud del pétalo son diferentes para las dos especies.

De hecho, las medias de cada una de la dos muestras son 1.4075 y 4.19,
valores muy diferentes.
\end{frame}

\begin{frame}[fragile]{Ejemplo}
\protect\hypertarget{ejemplo-4}{}
El intervalo de confianza al 95\% de confianza para la diferencia de
medias \(\mu_{{setosa}}-\mu_{{versicolor}}\) asociado al contraste
anterior vale, si nos fijamos en el ``output'' del \texttt{t.test}:

\begin{Shaded}
\begin{Highlighting}[]
\KeywordTok{t.test}\NormalTok{(muestra.setosa}\OperatorTok{$}\NormalTok{Petal.Length,muestra.versicolor}\OperatorTok{$}\NormalTok{Petal.Length,}
       \DataTypeTok{alternative=}\StringTok{"two.sided"}\NormalTok{)}\OperatorTok{$}\NormalTok{conf.int}
\end{Highlighting}
\end{Shaded}

\begin{verbatim}
## [1] -2.913186 -2.651814
## attr(,"conf.level")
## [1] 0.95
\end{verbatim}

intervalo que no contiene el valor cero y está totalmente a la izquierda
de cero. Por tanto, debemos rechazar la hipótesis nula.
\end{frame}

\begin{frame}[fragile]{Ejemplo}
\protect\hypertarget{ejemplo-5}{}
Fijémonos que hemos considerado que las varianzas de las dos variables
son diferentes. Si las hubiésemos considerado iguales, tendríamos que
hacer:

\begin{Shaded}
\begin{Highlighting}[]
\KeywordTok{t.test}\NormalTok{(muestra.setosa}\OperatorTok{$}\NormalTok{Petal.Length,muestra.versicolor}\OperatorTok{$}\NormalTok{Petal.Length,}
       \DataTypeTok{alternative=}\StringTok{"two.sided"}\NormalTok{,}\DataTypeTok{var.equal =} \OtherTok{TRUE}\NormalTok{)}
\end{Highlighting}
\end{Shaded}

\begin{verbatim}
## 
##  Two Sample t-test
## 
## data:  muestra.setosa$Petal.Length and muestra.versicolor$Petal.Length
## t = -42.766, df = 78, p-value < 2.2e-16
## alternative hypothesis: true difference in means is not equal to 0
## 95 percent confidence interval:
##  -2.91203 -2.65297
## sample estimates:
## mean of x mean of y 
##    1.4075    4.1900
\end{verbatim}
\end{frame}

\begin{frame}[fragile]{Ejemplo}
\protect\hypertarget{ejemplo-6}{}
En este caso, el p-valor también es despreciable, por lo que llegamos a
la misma conclusión anterior: las medias son diferentes.

Más adelante veremos cómo realizar un contraste de varianzas para
comprobar si éstas son iguales o no y por tanto, actuar en consecuencia
con el parámetro \texttt{var.equal}.
\end{frame}

\begin{frame}{Contrastes para dos proporciones \(p_1\) y \(p_2\)}
\protect\hypertarget{contrastes-para-dos-proporciones-p_1-y-p_2}{}
\end{frame}

\begin{frame}{Test de Fisher}
\protect\hypertarget{test-de-fisher}{}
Tenemos dos variables aleatorias \(X_1\) y \(X_2\) Bernoulli de
proporciones \(p_1\) y \(p_2\)

Tomamos m.a.s. de cada una y obtenemos la tabla siguiente:

\begin{longtable}[]{@{}llll@{}}
\toprule
& \(X_1\) & \(X_2\) & Total\tabularnewline
\midrule
\endhead
Éxitos & \(n_{11}\) & \(n_{12}\) & \(n_{1\bullet}\)\tabularnewline
Fracasos & \(n_{21}\) & \(n_{22}\) & \(n_{2\bullet}\)\tabularnewline
Total & \(n_{\bullet 1}\) & \(n_{\bullet 2}\) &
\(n_{\bullet\bullet}\)\tabularnewline
\bottomrule
\end{longtable}
\end{frame}

\begin{frame}{Test de Fisher}
\protect\hypertarget{test-de-fisher-1}{}
donde \(n_{11}\) es la cantidad de éxitos en la primera muestra,
\(n_{12}\), la cantidad de éxitos en la segunda muestra, \(n_{21}\), la
cantidad de fracasos en la primera muestra y \(n_{22}\), la cantidad de
fracasos en la segunda muestra.

De la misma forma, \(n_{1\bullet}\), es la cantidad total de éxitos en
las dos muestras y \(n_{2\bullet}\) la cantidad total de fracasos en las
dos muestras.

Por último, \(n_{\bullet 1}\) es el tamaño de la primera muestra,
\(n_{\bullet 2}\), el tamaño de la segunda muestra y
\(n_{\bullet\bullet}=n_{\bullet 1}+n_{\bullet 2}\) es la suma de los dos
tamaños.
\end{frame}

\begin{frame}{Test de Fisher}
\protect\hypertarget{test-de-fisher-2}{}
Supongamos \(p_1=p_2\).

Para hallar la probabilidad de obtener \(n_{11}\) éxitos para la
variable \(X_1\) podemos razonar de la forma siguiente:

En una bolsa tenemos \(n_{1\bullet}\) bolas E y \(n_{2\bullet}\) bolas
F. La probabilidad anterior sería la probabilidad de obtener \(n_{11}\)
bolas E si escogemos \(n_{\bullet 1}\) de golpe.

Sea \(X\) una variable hipergeométrica de parámetros
\(H(n_{1\bullet},n_{2\bullet},n_{\bullet1})\). La probabilidad anterior
sería: \(P(X=n_{11})\).

Usaremos la variable anterior \(X\) como estadístico de contraste.
\end{frame}

\begin{frame}{Test de Fisher}
\protect\hypertarget{test-de-fisher-3}{}
Nos planteamos los contrastes siguientes:

\(\left\{\begin{array}{l} H_0:p_1=p_2,\\ H_1:p_1> p_2. \end{array}\right.\)

\(\left\{\begin{array}{l} H_0:p_1=p_2,\\ H_1:p_1< p_2. \end{array}\right.\)

\(\left\{\begin{array}{l} H_0:p_1=p_2,\\ H_1:p_1\neq p_2. \end{array}\right.\)
\end{frame}

\begin{frame}{Test de Fisher}
\protect\hypertarget{test-de-fisher-4}{}
Los \textbf{p-valores} serán los siguientes:

\(p\)-valor:
\(P(H(n_{1\bullet},n_{2\bullet},n_{\bullet1})\geq n_{11})\).

\(p\)-valor:
\(P(H(n_{1\bullet},n_{2\bullet},n_{\bullet1})\leq n_{11})\).

\(p\)-valor: \(2\min\{P(H\leq n_{11}), P(H\geq n_{11})\}\).
\end{frame}

\begin{frame}{Ejemplo}
\protect\hypertarget{ejemplo-7}{}
\textbf{Ejemplo}

Para determinar si el Síndrome de Muerte Repentina del Bebé (SIDS) tiene
componiendo genético, se consideran los casos de SIDS en parejas de
gemelos monocigóticos y dicigóticos. Sea:

\begin{itemize}
\item
  \(p_1\): proporción de parejas de gemelos monocigóticos con algún caso
  de SIDS donde solo un hermano la sufrió.
\item
  \(p_2\): proporción de parejas de gemelos dicigóticos con algún caso
  de SIDS donde solo un hermano la sufrió.
\end{itemize}

Si el SIDS tiene componiendo genético, es de esperar que \(p_1<p_2\).

Nos piden realizar el contraste siguiente: \[
\left\{\begin{array}{l}
H_0:p_1=p_2,\\
H_1:p_1< p_2.
\end{array}\right.
\]
\end{frame}

\begin{frame}{Ejemplo}
\protect\hypertarget{ejemplo-8}{}
En un estudio (\emph{Peterson et al, 1980}), se obtuvieron los datos
siguientes:

\begin{longtable}[]{@{}llll@{}}
\toprule
Casos de SIDS & Monocigóticos & Dicigóticos & Total\tabularnewline
\midrule
\endhead
Uno & 23 & 35 & 58\tabularnewline
Dos & 1 & 2 & 3\tabularnewline
Total & 24 & 37 & 61\tabularnewline
\bottomrule
\end{longtable}
\end{frame}

\begin{frame}[fragile]{Ejemplo}
\protect\hypertarget{ejemplo-9}{}
El \textbf{p-valor} del contraste anterior sería:
\(P(H(58,3,24)\leq 23)\):

\begin{Shaded}
\begin{Highlighting}[]
\KeywordTok{phyper}\NormalTok{(}\DecValTok{23}\NormalTok{,}\DecValTok{58}\NormalTok{,}\DecValTok{3}\NormalTok{,}\DecValTok{24}\NormalTok{)}
\end{Highlighting}
\end{Shaded}

\begin{verbatim}
## [1] 0.7841067
\end{verbatim}

Al obtener un \(p\)-valor grande, podemos concluir que no tenemos
evidencias suficientes para rechazar la hipótesis nula y por tanto, el
SID no tiene componente genética.
\end{frame}

\begin{frame}[fragile]{Test de Fisher en \texttt{R}}
\protect\hypertarget{test-de-fisher-en-r}{}
\begin{itemize}
\tightlist
\item
  El test exacto de Fisher está implementado en la función
  \texttt{fisher.test}. Su sintaxis es
\end{itemize}

\begin{Shaded}
\begin{Highlighting}[]
\KeywordTok{fisher.test}\NormalTok{(x, }\DataTypeTok{alternative=}\NormalTok{..., }\DataTypeTok{conf.level=}\NormalTok{...)}
\end{Highlighting}
\end{Shaded}

donde

\begin{itemize}
\tightlist
\item
  \texttt{x} es la matriz anterior, donde recordemos que los números de
  éxitos van en la primera fila y los de fracasos en la segunda, y las
  poblaciones se ordenan por columnas.
\end{itemize}
\end{frame}

\begin{frame}[fragile]{Test de Fisher en \texttt{R}. Ejemplo}
\protect\hypertarget{test-de-fisher-en-r.-ejemplo}{}
\textbf{Ejercicio}

Realicemos el contraste anterior de igualdad de proporciones de madres
fumadores de raza blanca y negra usando el test de Fisher.

En primer lugar calculamos las etiquetas de las madres de cada raza:

\begin{Shaded}
\begin{Highlighting}[]
\NormalTok{madres.raza.blanca =}\StringTok{ }\KeywordTok{rownames}\NormalTok{(birthwt[birthwt}\OperatorTok{$}\NormalTok{race}\OperatorTok{==}\DecValTok{1}\NormalTok{,])}
\NormalTok{madres.raza.negra =}\StringTok{ }\KeywordTok{rownames}\NormalTok{(birthwt[birthwt}\OperatorTok{$}\NormalTok{race}\OperatorTok{==}\DecValTok{2}\NormalTok{,])}
\end{Highlighting}
\end{Shaded}

Seguidamente, elegimos las muestras de tamaño 50 de cada raza y creamos
las muestras correspondientes:

\begin{Shaded}
\begin{Highlighting}[]
\KeywordTok{set.seed}\NormalTok{(}\DecValTok{2000}\NormalTok{)}
\NormalTok{madres.elegidas.blanca=}\KeywordTok{sample}\NormalTok{(madres.raza.blanca,}\DecValTok{50}\NormalTok{,}\DataTypeTok{replace=}\OtherTok{TRUE}\NormalTok{)}
\NormalTok{madres.elegidas.negra =}\StringTok{ }\KeywordTok{sample}\NormalTok{(madres.raza.negra,}\DecValTok{50}\NormalTok{, }\DataTypeTok{replace=}\OtherTok{TRUE}\NormalTok{)}
\NormalTok{muestra.madres.raza.blanca =}\StringTok{ }\NormalTok{birthwt[madres.elegidas.blanca,]}
\NormalTok{muestra.madres.raza.negra =}\StringTok{ }\NormalTok{birthwt[madres.elegidas.negra,]}
\end{Highlighting}
\end{Shaded}
\end{frame}

\begin{frame}[fragile]{Test de Fisher en \texttt{R}. Ejemplo}
\protect\hypertarget{test-de-fisher-en-r.-ejemplo-1}{}
Definimos ahora una nueva tabla de datos que contenga la información de
las dos muestras consideradas:

\begin{Shaded}
\begin{Highlighting}[]
\NormalTok{muestra.madres =}\StringTok{ }\KeywordTok{rbind}\NormalTok{(muestra.madres.raza.blanca,muestra.madres.raza.negra)}
\end{Highlighting}
\end{Shaded}

A continuación calculamos la matriz para usar en el test de Fisher:

\begin{Shaded}
\begin{Highlighting}[]
\NormalTok{(}\DataTypeTok{matriz.fisher=}\KeywordTok{table}\NormalTok{(muestra.madres}\OperatorTok{$}\NormalTok{smoke,muestra.madres}\OperatorTok{$}\NormalTok{race))}
\end{Highlighting}
\end{Shaded}

\begin{verbatim}
##    
##      1  2
##   0 24 33
##   1 26 17
\end{verbatim}
\end{frame}

\begin{frame}[fragile]{Test de Fisher en \texttt{R}. Ejemplo}
\protect\hypertarget{test-de-fisher-en-r.-ejemplo-2}{}
La matriz anterior no es correcta ya que la primera fila debería ser la
fila de ``éxitos'' y es la fila de ``fracasos''.

Lo arreglamos permutando las filas:

\begin{Shaded}
\begin{Highlighting}[]
\NormalTok{(}\DataTypeTok{matriz.fisher =} \KeywordTok{rbind}\NormalTok{(matriz.fisher[}\DecValTok{2}\NormalTok{,],matriz.fisher[}\DecValTok{1}\NormalTok{,]))}
\end{Highlighting}
\end{Shaded}

\begin{verbatim}
##       1  2
## [1,] 26 17
## [2,] 24 33
\end{verbatim}
\end{frame}

\begin{frame}[fragile]{Test de Fisher en \texttt{R}. Ejemplo}
\protect\hypertarget{test-de-fisher-en-r.-ejemplo-3}{}
Por último realizamos el contraste:

\begin{Shaded}
\begin{Highlighting}[]
\KeywordTok{fisher.test}\NormalTok{(matriz.fisher)}
\end{Highlighting}
\end{Shaded}

\begin{verbatim}
## 
##  Fisher's Exact Test for Count Data
## 
## data:  matriz.fisher
## p-value = 0.1056
## alternative hypothesis: true odds ratio is not equal to 1
## 95 percent confidence interval:
##  0.8723106 5.1038153
## sample estimates:
## odds ratio 
##   2.087041
\end{verbatim}
\end{frame}

\begin{frame}{Test de Fisher en \texttt{R}. Ejemplo}
\protect\hypertarget{test-de-fisher-en-r.-ejemplo-4}{}
El p-valor del contraste ha sido 0.1056, valor mayor que 0.1. Concluimos
que no tenemos evidencias para rechazar que las proporciones de madres
fumadoras de razas blanca y negra sean iguales.

O, dicho de otra manera, no rechazamos la hipótesis nula de igualdad de
proporciones.

\textbf{Ejercicio}

Como el test de Fisher es exacto, dejamos como ejercicio repetir el
experimento anterior pero en lugar de tomando muestras de tamaño 50,
tomando muestras de tamaño más pequeño como por ejemplo 10.

¡Atención!

Hay que ir con cuidado con la interpretación del intervalo de confianza
que da esta función: no es ni para la diferencia de las proporciones ni
para su cociente, sino para su \textbf{odds ratio}: el cociente \[
\Big({\frac{p_b}{1-p_b}}\Big)\Big/\Big({\frac{p_n}{1-p_n}}\Big).
\]
\end{frame}

\begin{frame}{Introducción a las \textbf{odds}}
\protect\hypertarget{introducciuxf3n-a-las-odds}{}
Odds

El \textbf{odds} de un suceso \(A\) es el cociente \[
\mbox{Odds}(A)=\frac{P(A)}{1-P(A)},
\] donde \(P(A)\) es la probabilidad que suceda \(A\) y mide cuántas
veces es más probable \(A\) que su contrario.

Las \emph{odds} son una función creciente de la probabilidad, y por lo
tanto \[
\mbox{Odds}(A)<\mbox{Odds}(B)\Longleftrightarrow P(A)<P(B).
\]
\end{frame}

\begin{frame}{Ejemplo}
\protect\hypertarget{ejemplo-10}{}
Esto permite comparar \emph{odds} en vez de probabilidades, con la misma
conclusión.

Por ejemplo, en nuestro caso, como el intervalo de confianza para la
\emph{odds ratio} va de 0.8723 a 5.1038. En particular, contiene el 1,
por lo que no podemos rechazar que

\[
\Big({\frac{p_b}{1-p_b}}\Big)\Big/\Big({\frac{p_n}{1-p_n}}\Big)=1,
\] es decir, no podemos rechazar que \[
\frac{p_b}{1-p_b}=\frac{p_n}{1-p_n}
\] y esto es equivalente a \(p_b=p_n\).
\end{frame}

\begin{frame}{Ejemplo}
\protect\hypertarget{ejemplo-11}{}
Si, por ejemplo, el intervalo de confianza hubiera ido de 0 a 0.8,
entonces la conclusión a este nivel de confianza hubiera sido que \[
\Big({\frac{p_b}{1-p_b}}\Big)\Big/\Big({\frac{p_n}{1-p_n}}\Big)<1
\] es decir, que \[
\frac{p_b}{1-p_b}<\frac{p_n}{1-p_n}
\] y esto es equivalente a \(p_b<p_n\).
\end{frame}

\begin{frame}{Contraste para dos proporciones: muestras grandes}
\protect\hypertarget{contraste-para-dos-proporciones-muestras-grandes}{}
Supongamos ahora que tenemos dos variables aleatorias \(X_1\) y \(X_2\)
de Bernoulli de parámetros \(p_1\) y \(p_2\).

Consideremos una m.a.s. de cada variable aleatoria de tamaños \(n_1\) y
\(n_2\), respectivamente, grandes (\(n_1,n_2\geq 50\) o 100): \[
\begin{array}{l}
X_{1,1}, X_{1,2},\ldots, X_{1,n_1},\mbox{ de }X_1,\\
X_{2,1}, X_{2,2},\ldots, X_{2,n_2},\mbox{ de }X_2.
\end{array}
\] Sean \(\widehat{p}_1\) y \(\widehat{p}_2\) sus proporciones
muestrales.

Suponemos que los números de éxitos y de fracasos en cada muestra son
\(\geq 5\) o 10).
\end{frame}

\begin{frame}{Contraste para dos proporciones: muestras grandes}
\protect\hypertarget{contraste-para-dos-proporciones-muestras-grandes-1}{}
Nos planteamos los contrastes siguientes como en el caso del test de
Fisher:

\(\left\{\begin{array}{l} H_0:p_1=p_2,\\ H_1:p_1> p_2. \end{array}\right.\)

\(\left\{\begin{array}{l} H_0:p_1=p_2,\\ H_1:p_1< p_2. \end{array}\right.\)

\(\left\{\begin{array}{l} H_0:p_1=p_2,\\ H_1:p_1\neq p_2. \end{array}\right.\)
\end{frame}

\begin{frame}{Contraste para dos proporciones: muestras grandes}
\protect\hypertarget{contraste-para-dos-proporciones-muestras-grandes-2}{}
El \textbf{estadístico de contraste} para los contrastes anteriores es:
\[Z=\frac{\widehat{p}_1 -\widehat{p}_2}{
\sqrt{\Big(\frac{n_1 \widehat{p}_1 +n_2 \widehat{p}_2}{n_1
+n_2}\Big)\Big(1-\frac{n_1 \widehat{p}_1 +n_2 \widehat{p}_2}{n_1
+n_2}\Big)\Big(\frac1{n_1}+\frac1{n_2}
\Big)}},\] que, usando el \emph{Teorema Central del Límite} y suponiendo
cierta la hipótesis nula \(H_0:p_1=p_2\), tiene aproximadamente una
distribución \(N(0,1)\).

Sea \(z_0\) el valor del \textbf{estadístico de contraste} usando las
proporciones muestrales \(\widehat{p}_1\) y \(\widehat{p}_2\).
\end{frame}

\begin{frame}{Contraste para dos proporciones: muestras grandes}
\protect\hypertarget{contraste-para-dos-proporciones-muestras-grandes-3}{}
Los \textbf{p-valores} serán los siguientes:

\(p\)-valor: \(P(Z\geq z_0)\).

\(p\)-valor: \(P(Z\leq z_0)\).

\(p\)-valor: \(2 P(Z \geq |z_0|)\).
\end{frame}

\begin{frame}{Ejemplo}
\protect\hypertarget{ejemplo-12}{}
\textbf{Ejercicio}

Se toman una muestra de ADN de 100 individuos con al menos tres
generaciones familiares en la isla de Mallorca, y otra de 50 individuos
con al menos tres generaciones familiares en la isla de Menorca.

Se quiere saber si un determinado alelo de un gen es presente con la
misma proporción en las dos poblaciones.

En la muestra mallorquina, 20 individuos lo tienen, y en la muestra
menorquina, 12.

Contrastar la hipótesis de igualdad de proporciones al nivel de
significación \(0.05\), y calcular el intervalo de confianza para la
diferencia de proporciones para este \(\alpha\).
\end{frame}

\begin{frame}{Ejemplo}
\protect\hypertarget{ejemplo-13}{}
Fijémonos que los tamaños de las muestras (100 y 50) son bastante
grandes

El contraste pedido es el siguiente: \[
\left\{\begin{array}{l}
H_0:p_1=p_2,\\
H_1:p_1\neq p_2,
\end{array}\right.
\] donde \(p_1\) y \(p_2\) representan las proporciones de individuos
que tienen el alelo en el gen para los individuos de la isla de Mallorca
y Menora, respectivamente.

El \textbf{estadístico de contraste} será:
\(Z=\frac{\widehat{p}_1 -\widehat{p}_2}{ \sqrt{\Big(\frac{n_1 \widehat{p}_1 +n_2 \widehat{p}_2}{n_1 +n_2}\Big)\Big(1-\frac{n_1 \widehat{p}_1 +n_2 \widehat{p}_2}{n_1 +n_2}\Big)\Big(\frac1{n_1}+\frac1{n_2} \Big)}}.\)

Las proporciones muestrales serán:
\(\widehat{p}_1 =\frac{20}{100}=0.2\),
\(\widehat{p}_2 = \frac{12}{50}=0.24\).

Si hallamos el valor que toma el \textbf{estadístico de contraste} para
las proporciones muestrales anteriores, obtenemos:
\[z_0=\frac{0.2 -0.24}{
\sqrt{\Big(\frac{20+12}{100+50}\Big)\Big(1-\frac{20 +12}{100+50}\Big)\Big(\frac1{100}+\frac1{50}\Big)}}=-0.564.\]
\end{frame}

\begin{frame}{Ejemplo}
\protect\hypertarget{ejemplo-14}{}
El \textbf{\(p\)-valor} será: \(2\cdot P(Z\geq |-0.564|)=0.573.\)

Decisión: como el \(p\)-valor es grande y mayor que \(\alpha=0.05\),
aceptamos la hipótesis que las dos proporciones son la misma al no tener
evidencias suficientes para rechazarla.

El intervalo de confianza para \(p_1-p_2\) al nivel de confianza
\((1-\alpha)\cdot 100\%\) en un contraste bilateral es \[
\begin{array}{l}
\left(\widehat{p}_1-\widehat{p}_2-z_{1-\frac{\alpha}2}\sqrt{\Big(\frac{n_1 \widehat{p}_1 +n_2 \widehat{p}_2}{n_1
+n_2}\Big)\Big(1-\frac{n_1 \widehat{p}_1 +n_2 \widehat{p}_2}{n_1
+n_2}\Big)\Big(\frac1{n_1}+\frac1{n_2}
\Big)},\right.\\
\quad
\left.\widehat{p}_1-\widehat{p}_2+z_{1-\frac{\alpha}2}\sqrt{\Big(\frac{n_1 \widehat{p}_1 +n_2 \widehat{p}_2}{n_1
+n_2}\Big)\Big(1-\frac{n_1 \widehat{p}_1 +n_2 \widehat{p}_2}{n_1
+n_2}\Big)\Big(\frac1{n_1}+\frac1{n_2}
\Big)}
\right)
\end{array}
\] que, en nuestro caso será:

\[
(0.2 -0.24-1.96\cdot 0.071, 0.2-0.24 +1.96\cdot 0.071) =(-0.179,0.099).
\] Observemos que contiene el 0. Por tanto no podemos rechazar que
\(p_1-p_2=0\) llegando a la misma conclusión que con el
\textbf{\(p\)-valor}.
\end{frame}

\begin{frame}[fragile]{Contrastes para dos proporciones en \texttt{R}}
\protect\hypertarget{contrastes-para-dos-proporciones-en-r}{}
\begin{itemize}
\tightlist
\item
  En \texttt{R} está implementado en la función \texttt{prop.test}, que
  además también sirve para contrastar dos proporciones por medio de
  muestras independientes grandes. Su sintaxis es
\end{itemize}

\begin{Shaded}
\begin{Highlighting}[]
\KeywordTok{prop.test}\NormalTok{(x, n, }\DataTypeTok{p =}\NormalTok{..., }\DataTypeTok{alternative=}\NormalTok{..., }\DataTypeTok{conf.level=}\NormalTok{...)}
\end{Highlighting}
\end{Shaded}

donde:

\begin{itemize}
\tightlist
\item
  \texttt{x} en el caso de un contraste de dos proporciones es un vector
  de dos números naturales cuyas componentes son los números de éxitos
  en las dos muestras.
\end{itemize}
\end{frame}

\begin{frame}[fragile]{Contrastes para proporciones en \texttt{R}}
\protect\hypertarget{contrastes-para-proporciones-en-r}{}
\begin{itemize}
\item
  Cuando estamos trabajando con dos muestras, \texttt{n} es el vector de
  dos entradas de sus tamaños.
\item
  El significado de \texttt{alternative} y \texttt{conf.level}, y sus
  posibles valores, son los usuales.
\end{itemize}
\end{frame}

\begin{frame}[fragile]{Contrastes para proporciones en \texttt{R}.
Ejemplo}
\protect\hypertarget{contrastes-para-proporciones-en-r.-ejemplo}{}
\textbf{Ejemplo}

Siguiendo el ejemplo anterior, contrastemos otra vez si la proporción de
madres fumadoras de raza blanca es la misma que la proporción de madres
fumadoras de raza negra pero usando ahora la función \texttt{prop.test}.

En primer lugar, calculamos cuántas madres fumadores hay de cada
muestra:

\begin{Shaded}
\begin{Highlighting}[]
\KeywordTok{table}\NormalTok{(muestra.madres.raza.blanca}\OperatorTok{$}\NormalTok{smoke)}
\end{Highlighting}
\end{Shaded}

\begin{verbatim}
## 
##  0  1 
## 24 26
\end{verbatim}

\begin{Shaded}
\begin{Highlighting}[]
\KeywordTok{table}\NormalTok{(muestra.madres.raza.negra}\OperatorTok{$}\NormalTok{smoke)}
\end{Highlighting}
\end{Shaded}

\begin{verbatim}
## 
##  0  1 
## 33 17
\end{verbatim}
\end{frame}

\begin{frame}[fragile]{Contrastes para proporciones en \texttt{R}.
Ejemplo}
\protect\hypertarget{contrastes-para-proporciones-en-r.-ejemplo-1}{}
\begin{Shaded}
\begin{Highlighting}[]
\NormalTok{n.blanca =}\StringTok{ }\KeywordTok{table}\NormalTok{(muestra.madres.raza.blanca}\OperatorTok{$}\NormalTok{smoke)[}\DecValTok{2}\NormalTok{] }\CommentTok{\# número de madres fumadoras }
\CommentTok{\# de raza blanca}
\NormalTok{n.negra =}\StringTok{ }\KeywordTok{table}\NormalTok{(muestra.madres.raza.negra}\OperatorTok{$}\NormalTok{smoke)[}\DecValTok{2}\NormalTok{] }\CommentTok{\# número de madres fumadoras }
\CommentTok{\# de raza negra}
\end{Highlighting}
\end{Shaded}

Tenemos un total de 26 madres fumadoras de raza blanca entre las 50 de
la muestra y 17 madres fumadores de raza negra entre las 50 de la
muestra.

Finalmente, realizamos el contraste planteado: \[
\left.
\begin{array}{ll}
H_0: & p_b = p_n, \\
H_1: & p_b \neq p_n,
\end{array}
\right\}
\] donde \(p_b\) y \(p_n\) representan las proporciones de madres
fumadoras de raza blanca y negra, respectivamente.
\end{frame}

\begin{frame}[fragile]{Contrastes para proporciones en \texttt{R}.
Ejemplo}
\protect\hypertarget{contrastes-para-proporciones-en-r.-ejemplo-2}{}
El contraste en \texttt{R} se realizaría de la forma siguiente:

\begin{Shaded}
\begin{Highlighting}[]
\KeywordTok{prop.test}\NormalTok{(}\KeywordTok{c}\NormalTok{(n.blanca,n.negra),}\KeywordTok{c}\NormalTok{(}\DecValTok{50}\NormalTok{,}\DecValTok{50}\NormalTok{))}
\end{Highlighting}
\end{Shaded}

\begin{verbatim}
## 
##  2-sample test for equality of proportions with continuity correction
## 
## data:  c(n.blanca, n.negra) out of c(50, 50)
## X-squared = 2.6112, df = 1, p-value = 0.1061
## alternative hypothesis: two.sided
## 95 percent confidence interval:
##  -0.03083246  0.39083246
## sample estimates:
## prop 1 prop 2 
##   0.52   0.34
\end{verbatim}
\end{frame}

\begin{frame}[fragile]{Contrastes para proporciones en \texttt{R}.
Ejemplo}
\protect\hypertarget{contrastes-para-proporciones-en-r.-ejemplo-3}{}
El p-valor del contraste ha sido 0.1061, muy parecido al del test de
Fisher, y mayor que 0.1. Concluimos otra vez que no tenemos evidencias
para rechazar que las proporciones de madres fumadoras de razas blanca y
negra sean iguales.

Si nos fijamos en el intervalo de confianza para la diferencia de
proporciones:

\begin{Shaded}
\begin{Highlighting}[]
\KeywordTok{prop.test}\NormalTok{(}\KeywordTok{c}\NormalTok{(n.blanca,n.negra),}\KeywordTok{c}\NormalTok{(}\DecValTok{50}\NormalTok{,}\DecValTok{50}\NormalTok{))}\OperatorTok{$}\NormalTok{conf.int}
\end{Highlighting}
\end{Shaded}

\begin{verbatim}
## [1] -0.03083246  0.39083246
## attr(,"conf.level")
## [1] 0.95
\end{verbatim}

vemos que el 0 está dentro de dicho intervalo, hecho que reafirma
nuestra conclusión.
\end{frame}

\begin{frame}{Contrastes de dos muestras más generales}
\protect\hypertarget{contrastes-de-dos-muestras-muxe1s-generales}{}
\end{frame}

\begin{frame}{Introducción}
\protect\hypertarget{introducciuxf3n}{}
Dado un parámetro \(\theta\) (\(\theta\) puede ser la media \(\mu\), la
proporción \(p\), etc.) y dadas dos poblaciones \(X_1\) y \(X_2\) cuyas
distribuciones dependen de parámetros \(\theta_1\) y \(\theta_2\), hemos
realizado contrastes en los que la hipótesis nula era de la forma
\(H_0:\theta_1 = \theta_2\), o \(H_0:\theta_1 - \theta_2=0\).

Existen contrastes más generales del tipo: \[
\left\{\begin{array}{l}
H_0:\theta_1-\theta_2=\Delta\\
H_1:\theta_1-\theta_2<\Delta\mbox{ o }\theta_1-\theta_2>\Delta\mbox{ o }\theta_1-\theta_2\neq\Delta
\end{array}\right.
\] con \(\Delta\in \mathbb{R}\).
\end{frame}

\begin{frame}{Cambios en los estadísticos de contraste}
\protect\hypertarget{cambios-en-los-estaduxedsticos-de-contraste}{}
Para realizar los contrastes anteriores, se pueden usar los mismos
\textbf{estadísticos} que en el caso en que
\(H_0:\theta_1 - \theta_2=0\) realizando los cambios siguientes:

\begin{itemize}
\item
  Si \(\theta =\mu\), la media, hay que sustituir
  \(\overline{X}_1-\overline{X}_2\) en el numerador del
  \textbf{estadístico} por \(\overline{X}_1-\overline{X}_2-\Delta\).
\item
  Si \(\theta =p\), proporción muestral, hay que sustituir
  \(\widehat{p}_1-\widehat{p}_2\) en el numerdor del
  \textbf{estadístico} por \(\widehat{p}_1-\widehat{p}_2-\Delta\).
\end{itemize}
\end{frame}

\begin{frame}{Ejemplo}
\protect\hypertarget{ejemplo-15}{}
\textbf{Ejemplo}

Tenemos dos tratamientos, A y B, de una dolencia. Tratamos 50 enfermos
con A y 100 con B. 20 enfermos tratados con A y 25 tratados con B
manifiestan haber sentido malestar general durante los 7 días
posteriores a iniciar el tratamiento.

¿Podemos concluir, a un nivel de significación del 5\%, que A produce
malestar general en una proporción de los enfermos que es 5 puntos
porcentuales superior a la proporción de los enfermos en que lo produce
B?
\end{frame}

\begin{frame}{Ejemplo}
\protect\hypertarget{ejemplo-16}{}
Sean \(p_1\) la proporción de enfermos en que A produce malestar general
y \(p_2\), la proporción de enfermos en que B produce malestar general.

El contraste a realizar es el siguiente: \[
\left\{\begin{array}{l}
H_0:p_1\leq p_2+0.05,\\
H_1:p_1>p_2+0.05.
\end{array}\right.
\]

El \textbf{estadístico de contraste} es el siguiente: \[
Z=\frac{\widehat{p}_1 -\widehat{p}_2-\Delta}{
\sqrt{\Big(\frac{n_1 \widehat{p}_1 +n_2 \widehat{p}_2}{n_1
+n_2}\Big)\Big(1-\frac{n_1 \widehat{p}_1 +n_2 \widehat{p}_2}{n_1
+n_2}\Big)\Big(\frac1{n_1}+\frac1{n_2}
\Big)}},
\] que, si la hipótesis nula es cierta, sigue aproximadamente la
distribución \(N(0,1)\).
\end{frame}

\begin{frame}{Ejemplo}
\protect\hypertarget{ejemplo-17}{}
Las proporciones y los tamaños muestrales son: \(\widehat{p}_1=0.4\),
\(\widehat{p}_2=0.25\), \(n_1=50\), \(n_2=100\) y el valor de \(\Delta\)
será \(\Delta=0.05\).

El valor que toma el \textbf{estadístico de contraste} es: \[
z_0=\frac{0.4-0.25-0.05}{
\sqrt{\Big(\frac{20+25}{50+100}\Big)\Big(1-\frac{20+25}{50+100}\Big)\Big(\frac1{50}+\frac1{100}\Big)}}=1.26.
\]

El \textbf{\(p\)-valor} del contraste será: \(P(Z\geq 1.26)= 0.104.\)

Decisión: como el \textbf{\(p\)-valor} es relativamente grande y mayor
que \(\alpha=0.05\), no tenemos indicios para rechazar la hipótesis que
\(p_1-p_2\) es inferior o igual a un \(5\%\).
\end{frame}

\begin{frame}{Ejemplo}
\protect\hypertarget{ejemplo-18}{}
Si hallamos el \textbf{intervalo de confianza} para \(p_1-p_2\) al nivel
de confianza \((1-\alpha)\cdot 100\%\), obtenemos:

\[
\begin{array}{l}
\left(\widehat{p}_1-\widehat{p}_2+z_{\alpha}\sqrt{\Big(\frac{n_1 \widehat{p}_1 +n_2 \widehat{p}_2}{n_1
+n_2}\Big)\Big(1-\frac{n_1 \widehat{p}_1 +n_2 \widehat{p}_2}{n_1
+n_2}\Big)\Big(\frac1{n_1}+\frac1{n_2}
\Big)},\infty
\right) =  \\
\left(0.4-0.25 -1.645\sqrt{\left(\frac{50 \cdot 0.4 +100\cdot 0.25}{50
+100}\right)\left(1-\frac{50 \cdot 0.4 +100\cdot 0.25}{50
+100}\right)\left(\frac1{50}+\frac1{100}\right)},\infty\right) = \\
(0.019,\infty)
\end{array}
\]

Nos fijamos que el intervalo anterior contiene el valor
\(\Delta =0.05\), razón que nos reafirma la decisión tomada de no
rechazar que \(p_1\leq p_2+ 0.05\) pero, en cambio, no contiene el valor
\(0\) y por tanto, podríamos rechazar que \(p_1=p_2\).
\end{frame}

\begin{frame}{Contrastes para dos varianzas}
\protect\hypertarget{contrastes-para-dos-varianzas}{}
\end{frame}

\begin{frame}{Introducción}
\protect\hypertarget{introducciuxf3n-1}{}
Dadas dos poblaciones de distribución normal e indpendientes, nos
planteamos si las varianzas de dichas poblaciones son iguales o
diferentes.

Una aplicación del contraste de varianzas es decidir qué opción elegir
en el marco de una comparación de medias de muestras independientes.

Tenemos dos variables aleatorias \(X_1\) y \(X_2\) normales de
desviaciones típicas \(\sigma_1\), \(\sigma_2\) desconocidas

Suponemos que tenemos una m.a.s de cada variable: \[
\begin{array}{l}
X_{1,1}, X_{1,2},\ldots, X_{1,n_1}\mbox{ de }X_1\\
X_{2,1}, X_{2,2},\ldots, X_{2,n_2}\mbox{ de }X_2
\end{array}
\] Sean \(\widetilde{S}_1^2\) y \(\widetilde{S}_2^2\) sus varianzas
muestrales.
\end{frame}

\begin{frame}{Contrastes planteados}
\protect\hypertarget{contrastes-planteados}{}
Nos planteamos los contrastes siguientes:

\(\left\{\begin{array}{l} H_0:\sigma_1=\sigma_2, \quad \left(\mbox{ o } H_0:\dfrac{\sigma_1^2}{\sigma_2^2}=1\right),\\ H_1:\sigma_1 > \sigma_2. \end{array} \right.\)

\(\left\{\begin{array}{l} H_0:\sigma_1=\sigma_2, \quad \left(\mbox{ o } H_0:\dfrac{\sigma_1^2}{\sigma_2^2}=1\right),\\ H_1:\sigma_1 < \sigma_2. \end{array} \right.\)

\(\left\{\begin{array}{l} H_0:\sigma_1=\sigma_2, \quad \left(\mbox{ o } H_0:\dfrac{\sigma_1^2}{\sigma_2^2}=1\right),\\ H_1:\sigma_1 \neq \sigma_2. \end{array} \right.\)
\end{frame}

\begin{frame}{Estadístico de contraste}
\protect\hypertarget{estaduxedstico-de-contraste}{}
Se emplea el siguiente \textbf{estadístico de contraste}: \[
F=\frac{\widetilde{S}_1^2}{\widetilde{S}_2^2}
\] que, si las dos poblaciones son normales y la hipótesis nula
\(H_0:\sigma_1=\sigma_2\) es cierta, tiene distribución \(F\) de Fisher
con grados de libertad \(n_1-1\) y \(n_2-1\).

Sea \(f_0\) el valor que toma usando las desviaciones típicas
muestrales.
\end{frame}

\begin{frame}{La distribución \(F\) de Fisher}
\protect\hypertarget{la-distribuciuxf3n-f-de-fisher}{}
La distribución \(F_{n,m}\) de Fisher, donde \(n,m\) son los grados de
libertad se define como el cociente de dos variables chi2 independientes
de \(n\) y \(m\) grados de libertad, respectivamente:
\({\chi_{n}^2}/{\chi_m^2}\).

Su función de densidad tiene la siguiente expresión: \[
f_{F_{n,m}}(x)=\frac{\Gamma\left(\frac{n+m}2\right)\cdot\left(\frac{m}{n}\right)^{m/2}x^{(m-2)/2}}
{\Gamma\left(\frac{n}2\right)\Gamma\left(\frac{m}2\right)\left(1+\frac{m}{n}x\right)^{(m+n)/2}},
\mbox{ si $x\geq 0$,}
\] donde \(\Gamma(x)=\int_0^{\infty} t^{x-1}e^{-t}\, dt,\) si \(x> 0\).

Se trata de una distribución no simétrica.
\end{frame}

\begin{frame}{La distribución \(F\) de Fisher}
\protect\hypertarget{la-distribuciuxf3n-f-de-fisher-1}{}
Gráfica de la función de densidad de algunas distribuciones \(F\) de
Fisher.

\includegraphics{contrastes_dos_muestras_files/figure-beamer/unnamed-chunk-24-1.pdf}
\end{frame}

\begin{frame}{p-valores}
\protect\hypertarget{p-valores}{}
Los \textbf{\(p\)-valores} asociados a los contrastes anteriores son:

\(p\)-valor: \(P(F_{n_1-1,n_2-1}\geq f_0)\).

\(p\)-valor: \(P(F_{n_1-1,n_2-1}\leq f_0)\).

\(p\)-valor:
\(\min\{2\cdot P(F_{n_1-1,n_2-1}\leq f_0),2\cdot P(F_{n_1-1,n_2-1}\geq f_0)\}\).
\end{frame}

\begin{frame}{Ejemplo}
\protect\hypertarget{ejemplo-19}{}
\textbf{Ejercicio}

Consideramos el ejemplo donde queríamos comparar los tiempos de
realización de una tarea entre estudiantes de dos grados \(G_1\) y
\(G_2\). Suponemos que estos tiempos siguen distribuciones normales.

Disponemos de dos muestras independientes de los tiempos usados por los
estudiantes de cada grado para realizar la tarea. Los tamaños de cada
muestra son \(n_1=n_2=40\).

Las desviaciones típicas muestrales de los tiempos empleados para cada
muestra son: \[
\widetilde{S}_1=1.201,\quad \widetilde{S}_2=1.579
\]

Contrastar la hipótesis de igualdad de varianzas al nivel de
significación \(0.05\).
\end{frame}

\begin{frame}{Ejemplo}
\protect\hypertarget{ejemplo-20}{}
El contraste planteado es el siguiente: \[
\left\{\begin{array}{l}
H_0:\sigma_1=\sigma_2,\\
H_1:\sigma_1\neq \sigma_2,
\end{array}\right.
\] donde \(\sigma_1\) y \(\sigma_2\) son las desviaciones típicas de los
tiempos empleados para realizar la tarea por los estudiantes de los
grados \(G_1\) y \(G_2\), respectivamente.

El \textbf{estadístico de contraste} para el contraste anterior es:
\(F=\frac{\widetilde{S}_1^2}{\widetilde{S}_2^2}\sim F_{39,39}\).

Dicho estadístico toma el siguiente valor:
\(f_0=\frac{1.201^2}{1.579^2}=0.579.\)
\end{frame}

\begin{frame}{Ejemplo}
\protect\hypertarget{ejemplo-21}{}
El \textbf{\(p\)-valor} para el contraste anterior será: \[
\begin{array}{l}
\min\{2\cdot P(F_{n_1-1,n_2-1}\leq f_0),2\cdot P(F_{n_1-1,n_2-1}\geq f_0)\}= \\
\min\{2\cdot P(F_{n_1-1,n_2-1}\leq 0.579),2\cdot P(F_{n_1-1,n_2-1}\geq 0.579)\} 
= \\ \min\{0.091,1.909\}=0.091.
\end{array}
\]

Decisión: como que el \(p\)-valor es moderado pero mayor que
\(\alpha=0.05\), no podemos rechazar la hipótesis que las dos varianzas
sean iguales.

Concluimos que no tenemos evidencias suficientes para rechazar que
\(\sigma_1= \sigma_2\).

Por tanto, en el contraste de las dos medias, tendríamos que suponer que
las varianzas de las dos poblaciones son la misma.
\end{frame}

\begin{frame}{Ejemplo}
\protect\hypertarget{ejemplo-22}{}
El \textbf{intervalo de confianza} para
\(\frac{\sigma_1^2}{\sigma_2^2}\) al nivel de confianza
\((1-\alpha)\cdot 100\%\) es \[
\left(\frac{\widetilde{S}_1^2}{\widetilde{S}_2^2}\cdot F_{n_1-1,n_2-1,\frac{\alpha}2},\frac{\widetilde{S}_1^2}{\widetilde{S}_2^2}\cdot F_{n_1-1,n_2-1,1-\frac{\alpha}2}\right) =
\left(\frac{1.201^2}{1.579^2}\cdot F_{39,39,0.025},\frac{1.201^2}{1.579^2}\cdot F_{39,39,0.975}\right)=(0.306,1.094)
\] Observemos que el intervalo de confianza anterior contiene el valor
1, hecho que reafirma la decisión tomada de no rechazar la hipótesis de
igualdad de varianzas.
\end{frame}

\begin{frame}{Ejemplo}
\protect\hypertarget{ejemplo-23}{}
\textbf{Ejemplo} Se desea comparar la actividad motora espontánea de un
grupo de 25 ratas control y otro de 36 ratas desnutridas. Se midió el
número de veces que pasaban ante una célula fotoeléctrica durante 24
horas. Los datos obtenidos fueron los siguientes:

\begin{longtable}[]{@{}llll@{}}
\toprule
& \(n\) & \(\overline{X}\) & \(\widetilde{S}\)\tabularnewline
\midrule
\endhead
1. Control & 25 & \(869.8\) & \(106.7\)\tabularnewline
2. Desnutridas & 36 & \(665\) & \(133.7\)\tabularnewline
\bottomrule
\end{longtable}

¿Se observan diferencias significativas entre el grupo de control y el
grupo desnutrido?

Supondremos que los datos anteriores provienen de poblaciones normales.
\end{frame}

\begin{frame}{Ejemplo}
\protect\hypertarget{ejemplo-24}{}
El contraste a realizar es el siguiente: \[
\left\{\begin{array}{l}
H_0:\mu_1=\mu_2,\\
H_1:\mu_1\neq \mu_2,
\end{array}\right.
\] donde \(\mu_1\) y \(\mu_2\) representan los valores medios del número
de veces que las ratas de control y desnutridas pasan ante la célula
fotoeléctrica, respectivamente.

Antes de nada, tenemos que averiguar si las varianzas de los dos grupos
son iguales o no ya que es un parámetro a usar en el contraste a
realizar.

Por tanto, en primer lugar, realizaremos el contraste: \[
\left\{\begin{array}{l}
H_0:\sigma_1=\sigma_2\\
H_1:\sigma_1\neq \sigma_2
\end{array}\right.
\] donde \(\sigma_1\) y \(\sigma_2\) representan las desviaciones
típicas del número de veces que las ratas de control y desnutridas pasan
ante la célula fotoeléctrica, respectivamente.
\end{frame}

\begin{frame}{Ejemplo}
\protect\hypertarget{ejemplo-25}{}
El \textbf{Estadístico de contraste} para el contraste anterior vale:
\(F=\frac{\widetilde{S}_1^2}{\widetilde{S}_2^2}\sim F_{24,35}.\)

El valor que toma es el siguiente:
\(f_0=\frac{106.7^2}{133.7^2}=0.637.\)

El \textbf{\(p\)-valor} para el contraste anterior vale: \[
\begin{array}{l}
\min\{2\cdot P(F_{n_1-1,n_2-1}\leq f_0),2\cdot P(F_{n_1-1,n_2-1}\geq f_0)\}= \\
\min\{2\cdot P(F_{n_1-1,n_2-1}\leq 0.637),2\cdot P(F_{n_1-1,n_2-1}\geq 0.637)\} 
= \\ \min\{0.251,1.749\}=0.251.
\end{array}
\] El \textbf{\(p\)-valor} es un valor grande, por tanto, concluimos que
no podemos rechazar la hipótesis nula y decidimos que las varianzas de
las dos poblaciones son iguales.
\end{frame}

\begin{frame}{Ejemplo}
\protect\hypertarget{ejemplo-26}{}
Realicemos a continuación el contraste pedido: \[
\left\{\begin{array}{l}
H_0:\mu_1=\mu_2\\
H_1:\mu_1\neq \mu_2
\end{array}\right.
\]

El \textbf{estadístico de contraste} al suponer que
\(\sigma_1= \sigma_2\), será:
\(T=\frac{\overline{X}_1-\overline{X}_2} {\sqrt{(\frac1{n_1}+\frac1{n_2})\cdot \frac{(n_1-1)\widetilde{S}_1^2+(n_2-1)\widetilde{S}_2^2} {n_1+n_2-2}}}\sim t_{59}\).

El valor que toma dicho estadístico en los valores muestrales vale:
\(t_0=\frac{869.8-665}{\sqrt{(\frac1{25}+\frac1{36})\cdot \frac{24\cdot 106.7^2+35\cdot 133.7^2} {25+36-2}}}=6.373.\)

El \textbf{\(p\)-valor} del contraste será:
\(p=2\cdot P(t_{59}\geq 6.373)\approx 0.\)

Decisión: como el \textbf{\(p\)-valor} es prácticamente nulo, concluimos
que tenemos evidencias suficientes para rechazar la hipótesis nula y por
tanto hay diferencias entre las ratas de control y las desnutridas entre
el número de veces que pasan ante la célula fotoeléctrica.
\end{frame}

\begin{frame}[fragile]{Contrastes para varianzas en \texttt{R}}
\protect\hypertarget{contrastes-para-varianzas-en-r}{}
La función para efectuar este test en \texttt{R} es \texttt{var.test}y
su sintaxis básica es la misma que la de \texttt{t.test} para dos
muestras:

\begin{Shaded}
\begin{Highlighting}[]
\KeywordTok{var.test}\NormalTok{(x, y, }\DataTypeTok{alternative=}\NormalTok{..., }\DataTypeTok{conf.level=}\NormalTok{...)}
\end{Highlighting}
\end{Shaded}

donde \texttt{x} e \texttt{y} son los dos vectores de datos, que se
pueden especificar mediante una fórmula como en el caso de
\texttt{t.test}, y el parámetro \texttt{alternative} puede tomar los
tres mismos valores que en los tests anteriores.
\end{frame}

\begin{frame}[fragile]{Contrastes para varianzas en \texttt{R}. Ejemplo}
\protect\hypertarget{contrastes-para-varianzas-en-r.-ejemplo}{}
\textbf{Ejercicio}

Recordemos que cuando explicábamos el contraste para dos medias
independientes, contrastamos si las medias de las longitudes del pétalo
para las especies setosa y versicolor eran iguales o no pero
necesitábamos saber si las varianzas eran iguales o no para poder
tenerlo en cuenta en la función \texttt{t.test}.

Veamos ahora si podemos considerar las varianzas iguales o no.

Las muestras eran \texttt{muestra.setosa} y \texttt{muestra.versicolor}.
\end{frame}

\begin{frame}[fragile]{Contrastes para varianzas en \texttt{R}. Ejemplo}
\protect\hypertarget{contrastes-para-varianzas-en-r.-ejemplo-1}{}
Realicemos el contraste de igualdad de varianzas:

\begin{Shaded}
\begin{Highlighting}[]
\KeywordTok{var.test}\NormalTok{(muestra.setosa}\OperatorTok{$}\NormalTok{Petal.Length,muestra.versicolor}\OperatorTok{$}\NormalTok{Petal.Length)}
\end{Highlighting}
\end{Shaded}

\begin{verbatim}
## 
##  F test to compare two variances
## 
## data:  muestra.setosa$Petal.Length and muestra.versicolor$Petal.Length
## F = 0.14331, num df = 39, denom df = 39, p-value = 1.738e-08
## alternative hypothesis: true ratio of variances is not equal to 1
## 95 percent confidence interval:
##  0.07579578 0.27095614
## sample estimates:
## ratio of variances 
##          0.1433085
\end{verbatim}
\end{frame}

\begin{frame}[fragile]{Contrastes para varianzas en \texttt{R}. Ejemplo}
\protect\hypertarget{contrastes-para-varianzas-en-r.-ejemplo-2}{}
El p-valor del contraste ha sido prácticamente cero. Por tanto,
concluimos que tenemos evidencias suficientes para afirmar que las
varianzas de las longitudes del pétalo de las flores de las especies
setosa y versicolor son diferentes.

Si nos fijamos en el intervalo de confianza en el cociente de varianzas
\(\frac{\sigma^2_{{ setosa}}}{\sigma^2_{{versicolor}}}\),

\begin{Shaded}
\begin{Highlighting}[]
\KeywordTok{var.test}\NormalTok{(muestra.setosa}\OperatorTok{$}\NormalTok{Petal.Length,muestra.versicolor}\OperatorTok{$}\NormalTok{Petal.Length)}\OperatorTok{$}\NormalTok{conf.int}
\end{Highlighting}
\end{Shaded}

\begin{verbatim}
## [1] 0.07579578 0.27095614
## attr(,"conf.level")
## [1] 0.95
\end{verbatim}

vemos que no contiene el valor 1, de hecho está a la izquierda de él.
Este hecho nos hace reafirmar la conclusión anterior.

Para que el contraste anterior tenga sentido, hemos de suponer que las
longitudes del pétalo de las flores de las especies setosa y versicolor
siguen distribuciones normales.
\end{frame}

\begin{frame}[fragile]{Contrastes para varianzas}
\protect\hypertarget{contrastes-para-varianzas}{}
\begin{itemize}
\item
  Hemos insistido en que el test F solo es válido si las dos poblaciones
  cuyas varianzas comparamos son normales.
\item
  ¿Qué podemos hacer si dudamos de su normalidad? Usar un test no
  paramétrico que no presuponga esta hipótesis.
\item
  Hay diversos tests no paramétricos para realizar contrastes
  bilaterales de dos varianzas. Aquí os recomendamos el \textbf{test de
  Fligner-Killeen}, implementado en la función \texttt{fligner.test}.

  \begin{itemize}
  \tightlist
  \item
    Se aplica o bien a una \texttt{list} formada por las dos muestras, o
    bien a una fórmula que separe un vector numérico en dos muestras por
    medio de un factor de dos niveles.
  \end{itemize}
\end{itemize}
\end{frame}

\begin{frame}[fragile]{Contrastes para varianzas. Ejemplo.}
\protect\hypertarget{contrastes-para-varianzas.-ejemplo.}{}
Realicemos el contraste previo de igualdad de varianzas usando el test
no paramétrico anterior para ver si llegamos a la misma conclusión:

\begin{Shaded}
\begin{Highlighting}[]
\KeywordTok{fligner.test}\NormalTok{(}\KeywordTok{list}\NormalTok{(muestra.setosa}\OperatorTok{$}\NormalTok{Petal.Length,muestra.versicolor}\OperatorTok{$}\NormalTok{Petal.Length))}
\end{Highlighting}
\end{Shaded}

\begin{verbatim}
## 
##  Fligner-Killeen test of homogeneity of variances
## 
## data:  list(muestra.setosa$Petal.Length, muestra.versicolor$Petal.Length)
## Fligner-Killeen:med chi-squared = 22.39, df = 1, p-value = 2.226e-06
\end{verbatim}

Como el p-valor vuelve a ser insignificante, llegamos a la misma
conclusión anterior: tenemos evidencias suficientes para afirmar que las
varianzas de las longitudes del pétalo de las flores de las especies
setosa y versicolor son diferentes.

La ventaja de este test es que no necesitamos la normalidad de las
muestras, aunque su potencia, que explicaremos más adelante, sea
inferior.
\end{frame}

\begin{frame}{Muestras emparejadas}
\protect\hypertarget{muestras-emparejadas}{}
\end{frame}

\begin{frame}{Introducción}
\protect\hypertarget{introducciuxf3n-2}{}
Las muestras consideradas hasta el momento se han supuesto
\textbf{independientes}.

Un caso completamente diferente es cuando las dos muestras corresponden
a los mismos individuos o a individuos emparejados por algún factor.

Ejemplos:

\begin{itemize}
\item
  Se estudia el estado de una dolencia a los mismos individuos antes y
  después de un tratamiento.
\item
  Se mide la incidencia de cáncer en parejas de hermanos gemelos.
\end{itemize}

En estos casos, se habla de \textbf{muestras emparejadas}, o
\textbf{paired samples} en inglés.
\end{frame}

\begin{frame}{Introducción}
\protect\hypertarget{introducciuxf3n-3}{}
Para decidir si hay diferencias entre los valores de dos
\textbf{muestras emparejadas}, el contraste más común consiste a
calcular las diferencias de los valores de cada una de las parejas de
muestras y realizar un contraste para averiguar si la media de las
diferencias es 0.

Observación: El \textbf{diseño experimental} para realizar un contraste
de \textbf{muestras emparejadas} se tiene que fijar \textbf{antes} de la
\textbf{recogida de datos}.
\end{frame}

\begin{frame}{Contrastes de medias de muestras emparejadas}
\protect\hypertarget{contrastes-de-medias-de-muestras-emparejadas}{}
En el caso de un contraste de muestras emparejadas, sean \(X_1\) y
\(X_2\) las variables correspondientes y sean \[
\begin{array}{l}
X_{1,1}, X_{1,2},\ldots, X_{1,n},\mbox{ de }X_1\\
X_{2,1}, X_{2,2},\ldots, X_{2,n},\mbox{ de }X_2
\end{array}
\] las m.a.s. de cada una de las variables correspondientes a las dos
muestras.

Fijémonos que, al ser la muestras emparejadas, los tamaños de las mismas
deben ser iguales.

Consideramos la variable diferencia \(D=X_1-X_2\). La m.a.s. de \(D\)
construida a partir de las muestras anteriores será: \[
D_1 =X_{1,1}-X_{2,1}, \ D_2=X_{1,2}-X_{2,2},\ldots, D_n=X_{1,n}-X_{2,n}.
\]
\end{frame}

\begin{frame}{Contrastes de medias de muestras emparejadas}
\protect\hypertarget{contrastes-de-medias-de-muestras-emparejadas-1}{}
Los contastes planteados son los siguientes:

\(\left\{\begin{array}{l} H_0:\mu_1=\mu_2,\\ H_1:\mu_1> \mu_2. \end{array}\right.\)

\(\left\{\begin{array}{l} H_0:\mu_1=\mu_2,\\ H_1:\mu_1< \mu_2. \end{array}\right.\)

\(\left\{\begin{array}{l} H_0:\mu_1=\mu_2,\\ H_1:\mu_1\neq \mu_2. \end{array}\right.\)
\end{frame}

\begin{frame}{Contrastes de medias de muestras emparejadas}
\protect\hypertarget{contrastes-de-medias-de-muestras-emparejadas-2}{}
que, escritos en términos de la media de la variable diferencia \(D\),
\(\mu_d\), serán:

\(\left\{\begin{array}{l} H_0:\mu_d=0,\\ H_1:\mu_d> 0. \end{array}\right.\)

\(\left\{\begin{array}{l} H_0:\mu_d=0,\\ H_1:\mu_d< 0. \end{array}\right.\)

\(\left\{\begin{array}{l} H_0:\mu_d=0,\\ H_1:\mu_d\neq 0. \end{array}\right.\)
\end{frame}

\begin{frame}{Contrastes de medias de muestras emparejadas}
\protect\hypertarget{contrastes-de-medias-de-muestras-emparejadas-3}{}
O sea, hemos reducido un contraste de medias de dos muestras
dependientes a un contraste de una sola media de una sola muestra.

A partir de aquí, podemos calcular los \textbf{p-valores} y los
\textbf{intervalos de confianza} de los contrates anteriores usando las
expresiones de los contrastes de una media de una sola media vistos
anteriormente.
\end{frame}

\begin{frame}{Ejemplo de medias emparejadas}
\protect\hypertarget{ejemplo-de-medias-emparejadas}{}
\textbf{Ejemplo de medias emparejadas}

Disponemos de dos algoritmos de alineamiento de proteínas. Los dos
producen resultados de la misma calidad.

Estamos interesados en saber cuál de los dos algoritmos es \emph{más
eficiente}, en el sentido de tener un tiempo de ejecución más corto.
Suponemos que dichos tiempos de ejecución siguen leyes normales.

Tomamos una muestra de proteínas y les aplicamos los dos algoritmos,
anotando los tiempos de ejecución sobre cada proteína.

Los resultados obtenidos son:

\begin{longtable}[]{@{}lllllllllll@{}}
\toprule
& 1 & 2 & 3 & 4 & 5 & 6 & 7 & 8 & 9 & 10\tabularnewline
\midrule
\endhead
algoritmo 1 & 8.1 & 11.9 & 11.4 & 12.9 & 9.0 & 7.2 & 12.4 & 6.9 & 8.9 &
8.3\tabularnewline
algoritmo 2 & 6.9 & 6.7 & 8.3 & 8.6 & 18.9 & 7.9 & 7.4 & 8.7 & 7.9 &
12.4\tabularnewline
diferencias & 1.2 & 5.2 & 3.1 & 4.3 & -9.9 & -0.7 & 5.0 & -1.8 & 1.0 &
-4.1\tabularnewline
\bottomrule
\end{longtable}
\end{frame}

\begin{frame}{Ejemplo de medias emparejadas}
\protect\hypertarget{ejemplo-de-medias-emparejadas-1}{}
La media y la desviación típica muestrales de las difencias son
\(\overline{d}=0.33,\) \(\tilde s_d = 4.715.\)

Queremos contrastar la igualdad de medias con el test que corresponda. Y
si son diferentes, decidir cuál tiene mayor tiempo de ejecución.

O sea, queremos realizar el contraste siguiente: \[
\left\{\begin{array}{l}
H_0:\mu_1=\mu_2,\\
H_1:\mu_1\neq \mu_2,
\end{array}\right.
\] donde \(\mu_1\) y \(\mu_2\) son los tiempos de ejecución de los
algoritmos 1 y 2, respectivamente.

Escribimos el contraste anterior en función de \(\mu_d\), la media de
las diferencias de los tiempos de ejecución entre los dos algoritmos: \[
\left\{\begin{array}{l}
H_0:\mu_d=0,\\
H_1:\mu_d\neq 0.
\end{array}\right.
\]
\end{frame}

\begin{frame}{Ejemplo de medias emparejadas}
\protect\hypertarget{ejemplo-de-medias-emparejadas-2}{}
El \textbf{estadístico de contraste} para el contraste anterior es
\(T=\frac{\overline{d}}{\widetilde{S}_d/\sqrt{n}},\) que tiene
distribución \(t_{n-1}=t_{9}\).

Dicho estadístico toma el siguiente valor usando los valores muestrales:
\(t_0=\frac{0.33}{4.715/\sqrt{10}}=0.221.\)

El \textbf{\(p\)-valor} del contraste anterior será:
\(p=2\cdot p(t_{9} > |0.221|) =0.83.\)

Es un valor grande. Por tanto, no tenemos evidencias suficientes para
rechazar la hipótesis nula y concluimos que los tiempos de ejecución de
los dos algoritmos es el mismo.
\end{frame}

\begin{frame}[fragile]{Contrastes para medias emparejadas en \texttt{R}.
El test t}
\protect\hypertarget{contrastes-para-medias-emparejadas-en-r.-el-test-t}{}
Recordemos la sintaxis básica del test t en \texttt{R}

\begin{Shaded}
\begin{Highlighting}[]
\KeywordTok{t.test}\NormalTok{(x, y, }\DataTypeTok{mu=}\NormalTok{..., }\DataTypeTok{alternative=}\NormalTok{..., }\DataTypeTok{conf.level=}\NormalTok{..., }\DataTypeTok{paired=}\NormalTok{..., }
       \DataTypeTok{var.equal=}\NormalTok{..., }\DataTypeTok{na.omit=}\NormalTok{...)}
\end{Highlighting}
\end{Shaded}

donde el único parámetro para indicarle si las muestras son emparejadas
o independientes es el parámetro \texttt{paired}: con
\texttt{paired=TRUE} indicamos que las muestras son emparejadas, y con
\texttt{paired=FALSE} (que es su valor por defecto) que son
independientes.
\end{frame}

\begin{frame}[fragile]{Ejemplo de dos muestras dependientes con
\texttt{R}}
\protect\hypertarget{ejemplo-de-dos-muestras-dependientes-con-r}{}
\textbf{Ejercicio}

Nos planteamos si la longitud del sépalo supera la longitud del pétalo
para las flores de la especie virginica en la tabla de datos
\texttt{iris}.

En este caso se trataría de un contraste de medias dependientes: \[
\left.
\begin{array}{ll}
H_0: & \mu_{{sépalo,virginica}} =\mu_{{pétalo,virginica}}, \\
H_1: & \mu_{{sépalo,virginica}} > \mu_{{pétalo,virginica}},
\end{array}
\right\}
\] donde \(\mu_{{sépalo,virginica}}\) y \(\mu_{{pétalo,virginica}}\) son
las longitudes del sépalo y del pétalo de las flores de la especie
virginica.
\end{frame}

\begin{frame}[fragile]{Ejemplo de dos muestras dependientes con
\texttt{R}}
\protect\hypertarget{ejemplo-de-dos-muestras-dependientes-con-r-1}{}
Para realizar dicho contraste, vamos a considerar una muestra de 40
flores de la especie virgínica y sobre \textbf{las mismas flores}
calcular las longitudes del sépalo y del pétalo.

En primer lugar seleccionamos las flores de la muestra:

\begin{Shaded}
\begin{Highlighting}[]
\KeywordTok{set.seed}\NormalTok{(}\DecValTok{100}\NormalTok{)}
\NormalTok{flores.elegidas.virginica=}\KeywordTok{sample}\NormalTok{(}\DecValTok{101}\OperatorTok{:}\DecValTok{150}\NormalTok{,}\DecValTok{40}\NormalTok{,}\DataTypeTok{replace=}\OtherTok{TRUE}\NormalTok{)}
\end{Highlighting}
\end{Shaded}

La muestra elegida será:

\begin{Shaded}
\begin{Highlighting}[]
\NormalTok{muestra.virginica =}\StringTok{ }\NormalTok{iris[flores.elegidas.virginica,]}
\end{Highlighting}
\end{Shaded}
\end{frame}

\begin{frame}[fragile]{Ejemplo de dos muestras dependientes con
\texttt{R}}
\protect\hypertarget{ejemplo-de-dos-muestras-dependientes-con-r-2}{}
El contraste a realizar es el siguiente:

\begin{Shaded}
\begin{Highlighting}[]
\KeywordTok{t.test}\NormalTok{(muestra.virginica}\OperatorTok{$}\NormalTok{Sepal.Length,muestra.virginica}\OperatorTok{$}\NormalTok{Petal.Length,}
       \DataTypeTok{paired=}\OtherTok{TRUE}\NormalTok{,}\DataTypeTok{alternative=}\StringTok{"greater"}\NormalTok{)}
\end{Highlighting}
\end{Shaded}

\begin{verbatim}
## 
##  Paired t-test
## 
## data:  muestra.virginica$Sepal.Length and muestra.virginica$Petal.Length
## t = 22.041, df = 39, p-value < 2.2e-16
## alternative hypothesis: true difference in means is greater than 0
## 95 percent confidence interval:
##  0.9050861       Inf
## sample estimates:
## mean of the differences 
##                    0.98
\end{verbatim}
\end{frame}

\begin{frame}{Ejemplo de dos muestras dependientes con \texttt{R}}
\protect\hypertarget{ejemplo-de-dos-muestras-dependientes-con-r-3}{}
Vemos que el p-valor del contraste es prácticamente nulo, lo que nos
hace concluir que tenemos evidencias suficientes para afirmar que la
longitud del sépalo es superior a la longitud del pétalo para las flores
de la especie virginica.

Fijémonos que la media de la diferencia entre las medias de las
longitudes del sépalo y del pétalo vale 0.98, valor suficientemente
alejado del cero para poder afirmar que la media de la longitud del
sépalo es superior a la media de la longitud del pétalo.
\end{frame}

\begin{frame}[fragile]{Ejemplo de dos muestras dependientes con
\texttt{R}}
\protect\hypertarget{ejemplo-de-dos-muestras-dependientes-con-r-4}{}
El intervalo de confianza al 95\% de confianza para la diferencia de
medias asociado al contraste anterior vale:

\begin{Shaded}
\begin{Highlighting}[]
\KeywordTok{t.test}\NormalTok{(muestra.virginica}\OperatorTok{$}\NormalTok{Sepal.Length,muestra.virginica}\OperatorTok{$}\NormalTok{Petal.Length,}
       \DataTypeTok{paired=}\OtherTok{TRUE}\NormalTok{,}\DataTypeTok{alternative=}\StringTok{"greater"}\NormalTok{)}\OperatorTok{$}\NormalTok{conf.int}
\end{Highlighting}
\end{Shaded}

\begin{verbatim}
## [1] 0.9050861       Inf
## attr(,"conf.level")
## [1] 0.95
\end{verbatim}

intervalo que no contiene el cero y que está a la derecha del mismo, lo
que nos hace reafirmar que tenemos evidencias suficientes para rechazar
la hipótesis nula \(H_0\).
\end{frame}

\begin{frame}{Contrastes de proporciones de muestras emparejadas}
\protect\hypertarget{contrastes-de-proporciones-de-muestras-emparejadas}{}
Supongamos que evaluamos dos características dicotómicas sobre una misma
muestra de \(n\) sujetos. Resumimos los resultados obtenidos en la tabla
siguiente:

\[
\begin{array}{r|c}
 & \ \mbox{Característica 1}\  \\
\mbox{Característica 2} &\ \ \, \mbox{Sí}\qquad \mbox{No}\\\hline
 \mbox{Sí} & \quad\ \  a \qquad \ \ \, b\quad  \\
 \mbox{No} & \quad\ \   c  \qquad \ \ \, d\quad
 \end{array}
\]
\end{frame}

\begin{frame}[fragile]{Contrastes para proporciones. Muestras
emparejadas}
\protect\hypertarget{contrastes-para-proporciones.-muestras-emparejadas}{}
Se cumple \(a+b+c+d=n\). Esta tabla quiere decir, naturalmente, que
\(a\) sujetos de la muestra tuvieron la característica 1 y la
característica 2, que \(b\) sujetos de la muestra tuvieron la
característica 2 y pero no tuvieron la característica 2, etc.

Vamos a llamar \(p_{1}\) a la proporción poblacional de individuos con
la característica 1, y \(p_{2}\) a la proporción poblacional de
individuos con la característica 2.

Queremos contrastar la hipótesis nula \(H_{0}:p_1=p_2\) contra alguna
hipótesis alternativa. En este caso, no pueden usarse las funciones
\texttt{prop.test} o \texttt{fisher.test}.
\end{frame}

\begin{frame}{Contrastes para proporciones. Muestras emparejadas}
\protect\hypertarget{contrastes-para-proporciones.-muestras-emparejadas-1}{}
La solución es realizar el contraste bilateral: (o los unilaterales
asociados) \[
\left\{\begin{array}{l}
H_{0}:p_1=p_2,\\
H_{1}:p_1\neq p_2.
\end{array}\right.
\] Dicho contraste tiene sentido cuando \(n\) es grande y el número
\(b+c\) de \textbf{casos discordantes} (en los que una característica da
Sí y la otra da No) es razonablemente grande, pongamos \(\geq 20\).
\end{frame}

\begin{frame}{Contrastes para proporciones. Muestras emparejadas}
\protect\hypertarget{contrastes-para-proporciones.-muestras-emparejadas-2}{}
El \textbf{estadístico de contraste} para el contraste anterior es
\(Z=\frac{\frac{b}{n}-\frac{c}{n}}{\sqrt{\frac{b+c}{n^2}}}\), cuya
distribución aproximada es una \(N(0,1)\). Sea \(z_0\) el valor que toma
sobre los valores muestrales.

Por tanto el \textbf{\(p\)-valor} será: \(p=2\cdot p(Z > |z_0|)\).

\textbf{Ejercicio}

Hallar los \textbf{\(p\)-valores} para los contrastes unilaterales.
\end{frame}

\begin{frame}{Ejemplo de proporciones emparejadas}
\protect\hypertarget{ejemplo-de-proporciones-emparejadas}{}
\textbf{Ejemplo de proporciones emparejadas}

Se toma una muestra de \(1000\) personas afectadas por migraña. Se les
facilita un fármaco porque aligere los síntomas.

Después de la administración se les pregunta si han notado alivio en el
dolor.

Al cabo de un tiempo se suministra a los mismos individuos un placebo y
se les vuelve a preguntar si han notado o no mejora.

Nos preguntamos si es más efectivo el fármaco que el placebo en base a
los resultados del estudio:

\begin{longtable}[]{@{}lll@{}}
\toprule
Fármaco/Placebo & Si & No\tabularnewline
\midrule
\endhead
Si & 300 & 62\tabularnewline
No & 38 & 600\tabularnewline
\bottomrule
\end{longtable}
\end{frame}

\begin{frame}{Ejemplo de proporciones emparejadas}
\protect\hypertarget{ejemplo-de-proporciones-emparejadas-1}{}
El contraste que nos piden realizar es el siguiente: \[
\left\{\begin{array}{l}
H_0:p_1=p_2\\
H_1:p_1> p_2
\end{array}\right.
\] donde \(p_1\) y \(p_2\) representan las proporciones de gente que
encuentra mejora con el fármaco y el placebo, respectivamente.

El estadístico de contraste para el contraste anterior es:
\(Z=\frac{\frac{b}{n}-\frac{c}{n}}{\sqrt{\frac{b+c}{n^2}}}\), cuya
distribución aproximada es una \(N(0,1)\), donde \(a=300\), \(b=62\),
\(c=38\) y \(d=600\) en nuestro caso.

El valor que toma dicho estadístico es:
\(z_0 = \frac{\frac{62}{1000}-\frac{38}{1000}}{\sqrt{\frac{62+38}{1000^2}}} =2.4.\)
\end{frame}

\begin{frame}{Ejemplo de proporciones emparejadas}
\protect\hypertarget{ejemplo-de-proporciones-emparejadas-2}{}
Este contraste solo es válido cuando la muestra es grande y el número de
\emph{casos discordantes} \(b+d\) (100 en nuestro caso) es ``bastante
grande'', \(\geq 20\).

El \textbf{\(p\)-valor} para el contraste considerado es
\(P(Z>2.4)=0.008,\) pequeño.

Por lo tanto, concluimos que tenemos evidencias suficientes para
rechazar la hipótesis nula y poder afirmar que el fármaco es más
efectivo que el placebo.
\end{frame}

\begin{frame}[fragile]{Contrastes para proporciones de muestras
emparejadas en \texttt{R}}
\protect\hypertarget{contrastes-para-proporciones-de-muestras-emparejadas-en-r}{}
En \texttt{R} podemos usar el \textbf{test de McNemar}, que se lleva a
cabo con la instrucción \texttt{mcnemar.test}. Su sintaxis básica es

\begin{Shaded}
\begin{Highlighting}[]
\KeywordTok{mcnemar.test}\NormalTok{(X)}
\end{Highlighting}
\end{Shaded}

donde \texttt{X} es la matriz
\(\left(\begin{array}{cc} a & b\\ c& d \end{array}\right)\) que
corresponde a la tabla anterior.
\end{frame}

\begin{frame}[fragile]{Contrastes para proporciones de muestras
emparejadas en \texttt{R}. Ejemplo}
\protect\hypertarget{contrastes-para-proporciones-de-muestras-emparejadas-en-r.-ejemplo}{}
\textbf{Ejercicio}

Usando la tabla de datos \textbf{birthw} del paquete \textbf{MASS},
vamos a ver si la proporción de madres fumadoras es la misma que la
proporción de madres hipertensas.

Para ello, vamos a considerar una muestra de 30 madres y vamos a
realizar el contraste correspondiente.

En primer lugar elegimos las madres y consideramos la muestra
correspondiente:

\begin{Shaded}
\begin{Highlighting}[]
\KeywordTok{set.seed}\NormalTok{(}\DecValTok{333}\NormalTok{)}
\NormalTok{madres.elegidas.prop.empar =}\StringTok{ }\KeywordTok{sample}\NormalTok{(}\DecValTok{1}\OperatorTok{:}\DecValTok{189}\NormalTok{,}\DecValTok{30}\NormalTok{,}\DataTypeTok{replace=}\OtherTok{TRUE}\NormalTok{)}
\NormalTok{muestra.madres.prop.empar =}\StringTok{ }\NormalTok{birthwt[madres.elegidas.prop.empar,]}
\end{Highlighting}
\end{Shaded}
\end{frame}

\begin{frame}[fragile]{Contrastes para proporciones de muestras
emparejadas en \texttt{R}. Ejemplo}
\protect\hypertarget{contrastes-para-proporciones-de-muestras-emparejadas-en-r.-ejemplo-1}{}
Seguidamente, calculamos la matriz para usar en el contraste:

\begin{Shaded}
\begin{Highlighting}[]
\NormalTok{(}\DataTypeTok{matriz.prop.empar =} \KeywordTok{table}\NormalTok{(muestra.madres.prop.empar}\OperatorTok{$}\NormalTok{smoke,muestra.madres.prop.empar}\OperatorTok{$}\NormalTok{ht))}
\end{Highlighting}
\end{Shaded}

\begin{verbatim}
##    
##      0  1
##   0 16  3
##   1 10  1
\end{verbatim}

Fijémonos que dicha matriz no es correcta ya que \(a=1\), \(b=10\),
\(c=3\) y \(d=16\). Arreglamos la matriz:

\begin{Shaded}
\begin{Highlighting}[]
\NormalTok{matriz.prop.empar =}\StringTok{ }\KeywordTok{rbind}\NormalTok{(matriz.prop.empar[}\DecValTok{2}\NormalTok{,],matriz.prop.empar[}\DecValTok{1}\NormalTok{,])}
\NormalTok{matriz.prop.empar =}\StringTok{ }\KeywordTok{cbind}\NormalTok{(matriz.prop.empar[,}\DecValTok{2}\NormalTok{],matriz.prop.empar[,}\DecValTok{1}\NormalTok{])}
\end{Highlighting}
\end{Shaded}
\end{frame}

\begin{frame}[fragile]{Contrastes para proporciones de muestras
emparejadas en \texttt{R}. Ejemplo}
\protect\hypertarget{contrastes-para-proporciones-de-muestras-emparejadas-en-r.-ejemplo-2}{}
Comprobamos que es correcta:

\begin{Shaded}
\begin{Highlighting}[]
\NormalTok{matriz.prop.empar}
\end{Highlighting}
\end{Shaded}

\begin{verbatim}
##      [,1] [,2]
## [1,]    1   10
## [2,]    3   16
\end{verbatim}
\end{frame}

\begin{frame}[fragile]{Contrastes para proporciones de muestras
emparejadas en \texttt{R}. Ejemplo}
\protect\hypertarget{contrastes-para-proporciones-de-muestras-emparejadas-en-r.-ejemplo-3}{}
Por último, realizamos el contraste planteado:

\begin{Shaded}
\begin{Highlighting}[]
\KeywordTok{mcnemar.test}\NormalTok{(matriz.prop.empar)}
\end{Highlighting}
\end{Shaded}

\begin{verbatim}
## 
##  McNemar's Chi-squared test with continuity correction
## 
## data:  matriz.prop.empar
## McNemar's chi-squared = 2.7692, df = 1, p-value = 0.09609
\end{verbatim}
\end{frame}

\begin{frame}{Contrastes para proporciones de muestras emparejadas en
\texttt{R}. Ejemplo}
\protect\hypertarget{contrastes-para-proporciones-de-muestras-emparejadas-en-r.-ejemplo-4}{}
Hemos obtenido un p-valor de 0.0961, valor que está entre 0.05 y 0.1, la
llamada zona de penumbra donde no se puede tomar una decisión clara.

Podemos decir, si consideramos que el p-valor es suficientemente grande,
que no tenemos evidencias suficientes para aceptar que la proporción de
madres fumadoras y con hipertensión sea diferente.

En otras palabras, no rechazamos la hipótesis nula \(H_0\).

Ahora bien, hay que tener en cuenta que el p-valor no es demasiado
grande para tal conclusión.
\end{frame}

\begin{frame}[fragile]{Contrastes para proporciones. Muestras
emparejadas.}
\protect\hypertarget{contrastes-para-proporciones.-muestras-emparejadas.}{}
Otra posibilidad para realizar un contraste de dos proporciones usando
muestras emparejadas, que no requiere de ninguna hipótesis sobre los
tamaños de las muestras, es usar de manera adecuada la función
\texttt{binom.test}.

Para explicar este método, consideremos la tabla siguiente, donde ahora
damos las probabilidades poblacionales de las cuatro combinaciones de
resultados: \[
\begin{array}{r|c}
 & \ \mbox{Característica 1}\  \\
\mbox{Característica 2} &\quad \ \!\mbox{Sí}\qquad\quad\, \mbox{No}\quad \\\hline
 \mbox{Sí} & \quad \  p_{11}  \qquad\quad p_{01}\quad  \\
 \mbox{No} & \quad \  p_{10} \qquad\quad  p_{00}\quad
 \end{array}
\]
\end{frame}

\begin{frame}{Contrastes para proporciones. Muestras emparejadas.}
\protect\hypertarget{contrastes-para-proporciones.-muestras-emparejadas.-1}{}
De esta manera \(p_1=p_{11}+p_{10}\) y \(p_2=p_{11}+p_{01}\).

Entonces, \(p_1=p_2\) es equivalente a \(p_{10}=p_{01}\) y cualquier
hipótesis alternativa se traduce en la misma desigualdad, pero para
\(p_{10}\) y \(p_{01}\):

\begin{itemize}
\item
  \(p_1\neq p_2\) es equivalente a \(p_{10}\neq p_{01}\);
\item
  \(p_1< p_2\) es equivalente a \(p_{10}< p_{01}\); y
\item
  \(p_1> p_2\) es equivalente a \(p_{10}> p_{01}\).
\end{itemize}

Por lo tanto podemos traducir el contraste sobre \(p_1\) y \(p_2\) al
mismo contraste sobre \(p_{10}\) y \(p_{01}\).
\end{frame}

\begin{frame}{Contrastes para proporciones. Muestras emparejadas.}
\protect\hypertarget{contrastes-para-proporciones.-muestras-emparejadas.-2}{}
La gracia ahora está en que si la hipótesis nula \(p_{10}=p_{01}\) es
cierta, entonces, en el total de casos discordantes, el número de
sujetos en los que la característica 1 da Sí y la característica 2 da No
sigue una ley binomial con \(p=0.5\).

Por lo tanto, podemos efectuar el contraste usando un test binomial
exacto tomando

\begin{itemize}
\item
  como muestra los casos discordantes de nuestra muestra, de tamaño
  \(b+c\),
\item
  como éxitos los sujetos que han dado Sí en la característica 1 y No en
  la característica 2, de tamaño \(c\),
\item
  con proporción a contrastar \(p=0.5\) y con hipótesis alternativa la
  que corresponda.
\end{itemize}
\end{frame}

\begin{frame}{Contrastes para proporciones. Muestras emparejadas.}
\protect\hypertarget{contrastes-para-proporciones.-muestras-emparejadas.-3}{}
La ventaja de este test es que su validez no requiere de ninguna
hipótesis sobre los tamaños de las muestras. El inconveniente es que el
intervalo de confianza que nos dará será para
\(p_{10}/(p_{10}+p_{01})\), y no permite obtener un intervalo de
confianza para la diferencia o el cociente de las probabilidades \(p_1\)
y \(p_2\) de interés.
\end{frame}

\begin{frame}[fragile]{Contrastes para proporciones de muestras
emparejadas en \texttt{R}. Ejemplo}
\protect\hypertarget{contrastes-para-proporciones-de-muestras-emparejadas-en-r.-ejemplo-5}{}
\textbf{Ejercicio}

Volvamos a realizar el contraste anterior usando este método.

Recordemos que la matriz de proporciones era:

\begin{Shaded}
\begin{Highlighting}[]
\NormalTok{matriz.prop.empar}
\end{Highlighting}
\end{Shaded}

\begin{verbatim}
##      [,1] [,2]
## [1,]    1   10
## [2,]    3   16
\end{verbatim}
\end{frame}

\begin{frame}[fragile]{Contrastes para proporciones de muestras
emparejadas en \texttt{R}. Ejemplo}
\protect\hypertarget{contrastes-para-proporciones-de-muestras-emparejadas-en-r.-ejemplo-6}{}
Por tanto, el tamaño de nuestra muestra será:

\begin{Shaded}
\begin{Highlighting}[]
\NormalTok{(}\DataTypeTok{n=}\NormalTok{matriz.prop.empar[}\DecValTok{1}\NormalTok{,}\DecValTok{2}\NormalTok{]}\OperatorTok{+}\NormalTok{matriz.prop.empar[}\DecValTok{2}\NormalTok{,}\DecValTok{1}\NormalTok{])}
\end{Highlighting}
\end{Shaded}

\begin{verbatim}
## [1] 13
\end{verbatim}

El número de éxitos será:

\begin{Shaded}
\begin{Highlighting}[]
\NormalTok{(é}\DataTypeTok{xitos=}\NormalTok{matriz.prop.empar[}\DecValTok{2}\NormalTok{,}\DecValTok{1}\NormalTok{])}
\end{Highlighting}
\end{Shaded}

\begin{verbatim}
## [1] 3
\end{verbatim}
\end{frame}

\begin{frame}[fragile]{Contrastes para proporciones de muestras
emparejadas en \texttt{R}. Ejemplo}
\protect\hypertarget{contrastes-para-proporciones-de-muestras-emparejadas-en-r.-ejemplo-7}{}
El contraste a realizar será:

\begin{Shaded}
\begin{Highlighting}[]
\KeywordTok{binom.test}\NormalTok{(éxitos,n,}\DataTypeTok{p=}\FloatTok{0.5}\NormalTok{)}
\end{Highlighting}
\end{Shaded}

\begin{verbatim}
## 
##  Exact binomial test
## 
## data:  éxitos and n
## number of successes = 3, number of trials = 13, p-value = 0.09229
## alternative hypothesis: true probability of success is not equal to 0.5
## 95 percent confidence interval:
##  0.05038107 0.53813154
## sample estimates:
## probability of success 
##              0.2307692
\end{verbatim}
\end{frame}

\begin{frame}{Contrastes para proporciones de muestras emparejadas en
\texttt{R}. Ejemplo}
\protect\hypertarget{contrastes-para-proporciones-de-muestras-emparejadas-en-r.-ejemplo-8}{}
Vemos que el p-valor es parecido usando el método anterior y por tanto,
las conclusiones son las mismas.
\end{frame}

\begin{frame}[fragile]{Guía rápida}
\protect\hypertarget{guuxeda-ruxe1pida}{}
Excepto en las que decimos lo contrario, todas las funciones para
realizar contrastes que damos a continuación admiten los parámetros
\texttt{alternative}, que sirve para especificar el tipo de contraste
(unilateral en un sentido u otro o bilateral), y \texttt{conf.level},
que sirve para indicar el nivel de confianza \(1-\alpha\). Sus valores
por defecto son contraste bilateral y nivel de confianza 0.95.
\end{frame}

\begin{frame}[fragile]{Guía rápida}
\protect\hypertarget{guuxeda-ruxe1pida-1}{}
\begin{itemize}
\item
  \texttt{t.test} realiza tests t para contrastar una o dos medias
  (tanto usando muestras independientes como emparejadas). Aparte de
  \texttt{alternative} y \texttt{conf.level}, sus parámetros principales
  son:

  \begin{itemize}
  \item
    \texttt{mu} para especificar el valor de la media que queremos
    contrastar en un test de una media.
  \item
    \texttt{paired} para indicar si en un contraste de dos medias usamos
    muestras independientes o emparejadas.
  \item
    \texttt{var.equal} para indicar en un contraste de dos medias usando
    muestras independientes si las varianzas poblacionales son iguales o
    diferentes.
  \end{itemize}
\item
  \texttt{sigma.test}, para realizar tests \(\chi^2\) para contrastar
  una varianza (o una desviación típica). Dispone de los parámetros
  \texttt{sigma} y \texttt{sigmasq} para indicar, respectivamente, la
  desviación típica o la varianza a contrastar.
\end{itemize}
\end{frame}

\begin{frame}[fragile]{Guía rápida}
\protect\hypertarget{guuxeda-ruxe1pida-2}{}
\begin{itemize}
\item
  \texttt{var.test}, para realizar tests F para contrastar dos varianzas
  (o dos desviaciones típicas).
\item
  \texttt{fligner.test}, para realizar tests no paramétricos de
  Fligner-Killeen para contrastar dos varianzas (o dos desviaciones
  típicas). No dispone de los parámetros \texttt{alternative} (solo
  sirve para contastes bilaterales) ni \texttt{conf.level} (no calcula
  intervalos de confianza).
\item
  \texttt{binom.test}, para realizar tests binomiales exactos para
  contrastar una proporción. Dispone del parámetro \texttt{p} para
  indicar la proporción a contrastar.
\item
  \texttt{prop.test}, para realizar tests aproximados para contrastar
  una proporción o dos proporciones de poblaciones usando muestras
  independientes. También dispone del parámetro \texttt{p} para indicar
  la proporción a contrastar en un contraste de una proporción.
\end{itemize}
\end{frame}

\begin{frame}[fragile]{Guía rápida}
\protect\hypertarget{guuxeda-ruxe1pida-3}{}
\begin{itemize}
\item
  \texttt{fisher.test}, para realizar tests exactos de Fisher para
  contrastar dos proporciones usando muestras independientes.
\item
  \texttt{mcnemar.test}, para realizar tests bilaterales de McNemar para
  contrastar dos proporciones usando muestras emparejadas. No dispone de
  los parámetros \texttt{alternative} ni \texttt{conf.level}.
\end{itemize}
\end{frame}

\end{document}
